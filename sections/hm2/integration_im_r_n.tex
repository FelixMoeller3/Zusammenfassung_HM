\section{Integration im $\mathbb{R}^n$}
\subsection{Satz von Fubini}
Es seien $p,q \in \mathbb{N}, n=p+q$. Es sei $I_1$ ein kompaktes Intervall im $\mathbb{R}^p$, $I_2$ sei ein kompaktes Intervall im $\mathbb{R}^q$,
es sei $I \coloneqq I_1 \times I_2 \subseteq \mathbb{R}^n$ und $f \in R(I)$.
\begin{enumerate} [a)]
    \item Für jedes feste $y \in I_2$ ist die Funktion $x \mapsto f(x,y)$ integrierbar über  $I_1$ und es sei 
    $g(y) \coloneqq \int_{I_1} f(x,y)\,dx$. Dann gilt $g \in R(I_2)$ und
    \begin{align*}
        \int_I f(x,y) d(x,y) = \int_{I_2} g(y)\,dy = \int_{I_2} (\int_{I_1} f(x,y)\,dx)\,dy
    \end{align*}
    \item Für jedes feste $x \in I_1$ ist die Funktion $y \mapsto f(x,y)$ integrierbar über  $I_2$ und es sei 
    $g(x) \coloneqq \int_{I_2} f(x,y)\,dy$. Dann gilt $g \in R(I_1)$ und
    \begin{align*}
        \int_I f(x,y) d(x,y) = \int_{I_1} g(x)\,dx = \int_{I_1} (\int_{I_2} f(x,y)\,dy)\,dx
    \end{align*}
\end{enumerate}
\underline{Folgerung:} Es sei $I = [a_1,b_1] \times \ldots \times [a_n,b_n]$ und $f \in C(I)$. Dann ist
\begin{align*}
    \int_I f(x)\,dx = \int_I f(x_1, \ldots, x_n) d(x_1, \ldots, x_n) = \int \limits_{a_1}^{b_1} (\ldots \int \limits_{a_{n-1}}^{b_{n-1}}(\int \limits_{a_n}^{b_n}
    f(x_1,\ldots,x_n)\,dx_n)\,dx_{n-1}\ldots)\,dx_1
\end{align*}
wobei die Reihenfolge der Integration \underline{beliebig vertauscht} werden kann.

\subsection{Prinzip von Cavalieri}
\label{sec: Cavalieri}
\underline{Voraussetzung:} Es sei $B \subseteq \mathbb{R}^{n+1}$ messbar. Für Punkte im $\mathbb{R}^{n+1}$ schreiben wir $(x,z)$ mit $x \in \mathbb{R}^n$ und $z \in \mathbb{R}$. 
Es seien $a,b \in \mathbb{R}$ so, dass $a \leq z \leq b \text{ } ((x,z) \in B)$. Für $z \in [a,b]$ sei 
\begin{align*}
    Q(z) \coloneqq \{ x \in \mathbb{R}^n: (x,z) \in B\}
\end{align*}
Weiter sei $Q(z)$ messbar für jedes $z \in [a,b]$. Dann ist $z \mapsto \lvert Q(z)\rvert$ integrierbar über $[a,b]$ und 
\begin{align*}
    \lvert B\rvert = \int \limits_a^b \lvert Q(z)\rvert dz
\end{align*}
\underline{Rotationskörper:} Sei $a < b, f \in R([a,b])$ und $f \geq 0$ auf $[a,b]$. Der Graph von $f$ rotiere z.B. um die $x$-Achse:
\begin{align*}
    B = \{ (x,y,z) \in \mathbb{R}^3: y^2+z^2 \leq f(x)^2\}
\end{align*}
Für $x \in [a,b]$ ist dann $Q(x) = \{ (y,z) \in \mathbb{R}^2: y^2+z^2\leq f(x)^2\}$. Also gilt: $\lvert Q(x) \rvert = \pi f(x)^2$ und somit $\lvert B \rvert = \pi \int \limits_a^b f(x)^2\,dx$ \\
\underline{Normalbereich bzgl. $x$-Achse:} \\
Sei $a,b \in \mathbb{R}, a < b, f,g \in C([a,b])$ und $f \leq g$ auf $[a,b]$. Dann heißt die Menge
\begin{align*}
    B \coloneqq \{ (x,y) \in \mathbb{R}^2: x \in [a,b], f(x) \leq y \leq g(x)\}
\end{align*}
ein \underline{Normalbereich bzgl. der $x$-Achse.} Es gilt:
\begin{align*}
    \int_B h(x,y) d(x,y) = \int_I h_B(x,y) d(x,y) = \int \limits_a^b (\int \limits_{f(x)}^{g(x)} h(x,y)\,dy)\,dx
\end{align*}
\underline{Normalbereich bzgl. der $y$-Achse:} \\
Seien $a,b,f$ und $g$ wie in obiger Definition. Dann heißt die Menge 
\begin{align*}
    B \coloneqq \{ (x,y) \in \mathbb{R}^3: y \in [a,b], f(y) \leq x \leq g(y)\}
\end{align*}
ein \underline{Normalbereich bzgl. der $y$-Achse}. Wie oben gilt:
\begin{align*}
    \int_B h(x,y) d(x,y) = \int \limits_a^b (\int \limits_{f(y)}^{g(y)} h(x,y)\,dx)\,dy
\end{align*}

\subsection{Substitutionsregel}
\label{sec: Sub}
Es sei $G \subseteq \mathbb{R}^n$ offen, $g \in C^1(G,\mathbb{R}^n)$ und $B \subseteq G$ kompakt und messbar. Weiter sei $g$ auf dem Inneren $B^\circ$ von $B$ injektiv
und det$g'(y) \neq 0 \text{ } (y \in B^\circ)$. Ist dann $A \coloneqq g(B)$ und $f \in C(A,\mathbb{R)}$, so ist $A$ kompakt und messbar und es gilt:
\begin{align*}
    \int_A f(x)\,dx = \int_B f(g(y)) \lvert \text{det} g'(y)\rvert\,dy
\end{align*}
\underline{Polarkoordinaten (n=2):} \\
$x = r\cos \varphi, y = r\sin \varphi \text{ } (r = \lVert(x,y)\rVert = \sqrt{x^2 + y^2})$
\begin{align*}
    g(r,\varphi) \coloneqq (r\cos \varphi, r\sin \varphi), \det g'(r,\varphi) = r \text{ } ((r,\varphi) \in G \coloneqq \mathbb{R}^2)
\end{align*}
(Typische Menge: $A=\{ (x,y) \in \mathbb{R}^2: 0 \leq x^2+y^2\leq R^2\} (R \in \mathbb{R})$) \\
Betrachte $0 \leq \varphi_1 < \varphi_2 \leq 2\pi, 0 \leq R_1 < R_2$
\begin{align*}
    A \coloneqq \{ (r\cos \varphi, r\sin \varphi): \varphi \in [\varphi_1,\varphi_2], r \in [R_1,R_2]\}
\end{align*}
Mit $B \coloneqq [R_1,R_2] \times [\varphi_1, \varphi_2]$ ist $A = g(B)$. Auf $B^\circ = (R_1,R_2) \times (\varphi_1,\varphi_2)$ ist $g$ injektiv und $\det g' \neq 0$.
Ist nun $f \in C(A,\mathbb{R})$, so gilt: 
\begin{align*}
    \int_A f(x,y) d(x,y) = \int_B f(r \cos \varphi,r \sin \varphi) \cdot \underbrace{r}_{= \lvert \det g'(r,\varphi)\rvert} d(r, \varphi) \stackrel{\text{Fubini}}{=}
    \int \limits_{\varphi_1}^{\varphi_2} (\int \limits_{R_1}^{R_2} f(r \cos \varphi, r \sin \varphi)r dr) d\varphi
\end{align*}
\underline{Zylinderkoordinaten (n=3):} \\
\begin{align*}
    \begin{rcases} 
        x = r \cos \varphi \\
        y = r \sin \varphi \\
        z = z \\
    \end{rcases} g(r,\varphi,z) \coloneqq (r \cos \varphi,r \sin \varphi, z) \det g'(r,\varphi,z) = r
\end{align*}
(Typische Menge: $ \{ (x,y,z) \in \mathbb{R}^3: x^2+y^2 \leq R^2, 0 \leq z \leq h\} \text{ } (R,h \in \mathbb{R}_+)$)\\
Es seien $A,B \subseteq \mathbb{R}^3$ wie bei der \hyperref[sec: Sub]{Substitutionsregel} und $f \in C(A,\mathbb{R})$. Dann gilt: 
\begin{align*}
    \int_A f(x,y,z) d(x,y,z) = \int_B f(r \cos \varphi, r \sin \varphi, z) \cdot r d(r,\varphi,z)
\end{align*}
\underline{Kugelkoordinaten (n=3):} \\
Für $\varphi = [0,2\pi], \vartheta \in [-\frac{\pi}{2}, \frac{\pi}{2}], r=\lVert (x,y,z)\rVert = \sqrt{x^2+y^2+z^2}, x= r\cos \varphi \cos \vartheta, y=r \sin \varphi \cos \vartheta$,
$z=r \sin \vartheta$,
\begin{align*}
    g(r,\varphi,\vartheta) \coloneqq (r\cos \varphi \cos \varphi, r \sin \varphi \cos \vartheta, r \sin \vartheta), \lvert \det g'(r,\varphi,\vartheta)\rvert = r^2 \cos \vartheta
\end{align*}
Sind $A,B \subseteq \mathbb{R}^3$ wie beim \hyperref[sec: Cavalieri]{Prinzip von Cavalieri} (also $A=g(B)$), so gilt für $f \in C(A,\mathbb{R})$:
\begin{align*}
    \int_A f(x,y,z) d(x,y,z) = \int_B f(g(r,\varphi,\vartheta)) \cdot r^2 \cos \vartheta d(r,\varphi,\vartheta)
\end{align*}
Typische Menge: $\{ (x,y,z) \in \mathbb{R}^3: x,y,z \geq 0, x^2+y^2+z^2 \leq 1\}$