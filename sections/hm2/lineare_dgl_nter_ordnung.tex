\section{Lineare Dgl n-ter Ordnung mit konstanten Koeffizienten}
Sei 
\begin{align*}
    Ly \coloneqq y^{(n)} + a_{n-1}y^{(n-1)}+ \ldots + a_1y' + a_0y
\end{align*}
Die Dgl $Ly(x)=b(x)$ heißt lineare Dgl n-ter Ordnung mit konstanten Koeffizienten. \\
Lösungsmethode für die homogene Gleichung ($Ly(x)=0$):
\begin{enumerate}
    \item $p(\lambda)=\lambda^n + a_{n-1}\lambda^{n-1} + \ldots + a_1\lambda + a_0$ heißt charakteristisches Polynom der homogenen Dgl.
    Bestimme dessen Nullstellen.
    \item Fasse alle reellen Nullstellen und alle komplexen ohne ihr komplex Konjugiertes in einer Menge $M$ zusammen.
    \item FÜr jedes $\lambda_j$ aus $M$ mit Vielfachheit $k_j$: \\
    Fall 1: Ist $\lambda_j$ reell, so sind
    \begin{align*}
        e^{\lambda_jx}, xe^{\lambda_jx}, \ldots, x^{k_j-1}e^{\lambda_jx}
    \end{align*}
    $k_j$ linear unabhängige Lösungen der Dgl. \\
    Fall 2: $\lambda_j = \alpha+i\beta \in \mathbb{C} \setminus \mathbb{R}$. Dann sind
    \begin{align*}
        e^{\alpha x}\cos \beta x, \ldots, x^{kj-1}e^{\alpha x}\cos \beta x \\
        e^{\alpha x}\sin \beta x, \ldots, x^{kj-1}e^{\alpha x}\sin \beta x
    \end{align*}
    $2k_j$ linear unabhängige Lösungen der Dgl
    \item Führt man 3. für alle Elemente aus $M$ durch, so erhält man ein Fundamentalsystem zur Dgl.
\end{enumerate}
\underline{Lösung der inhomogenen Gleichung ($Ly(x)=b(x)$:} \\
Seien $\gamma,\delta \in \mathbb{R}, m \in \mathbb{N}_0,q$ ein Polynom vom Grad m und $b$ von der Form $b(x)=q(x)e^{\gamma x}\cos(\delta x)$
oder $b(x)=q(x)e^{\gamma x}\sin(\delta x)$. Sei $p$ das charakteristische Polynom von $Ly(x)=0$. \\
Fall 1: $p(\gamma + i\delta) \neq 0$. Ansatz:
\begin{align*}
    y_p(x) \coloneqq (\widehat{q}(x)\cos(\delta x) + \Tilde{q}(x)\sin(\delta x))e^{\gamma x}
\end{align*}
Fall 2: $\gamma + i \delta$ ist eine $v$-fache Nullstelle von $p$. Ansatz:
\begin{align*}
    y_p(x) \coloneqq x^v(\widehat{q}(x)\cos(\delta x) + \Tilde{q}(x) \sin(\delta x))e^{\gamma x}
\end{align*}
$\widehat{q}$ und $\Tilde{q}$ haben dabei den gleichen Grad wie $q(x)$ oben. \\
Die allgemeine Lösung ist also
\begin{align*}
    y(x) = c_1m_1 + \ldots + c_nm_n + y_p(x)
\end{align*}
für $m_1, \ldots, m_n$ aus dem Fundamentalsystem.