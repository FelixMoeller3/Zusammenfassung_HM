\section{Konvergenz im $\mathbb{R}^n$}
Eine Folge $(a^{(k)})$ heißt konvergent $\Leftrightarrow \exists a \in \mathbb{R}^n: \lVert a^{(k)} - a \rVert \to 0 \text{ } (k \to \infty)$
Ist $a=(a_1, \ldots, a_n) \in \mathbb{R}^n$, so gilt: $a^{(k)} \to a \text{ } (k \to \infty) \Leftrightarrow \forall j \in \{ 1, \ldots, n\}:$
$a_j^{(k)} \to a_j \text{ } (k \to \infty)$ Konvergenz im $\mathbb{R}^n$ ist also gleichbedeutend mit \underline{koordinatenweiser Konvergenz}.
Sonst vieles wie in HM1 (Bolzano-Weierstraß, Cauchy-Kriterium,...)
