\section{Fouriertransformation}
\underline{Definition:} Eine Funktion heißt auf $[a,b]$ stückweise stetig/glatt $\Leftrightarrow$ Es gibt eine Zerlegung $t_0, \ldots, t_m \in [a,b]$,
sodass $a=t_0 < t_1 < \ldots < t_m=b, \text{ } g \in C((t_{j-1},t_j)), j \in \{ 1, \ldots,m\}$ und es existieren
\begin{enumerate}
    \item $g(a+),g(b-),g(t_j\pm) \text{ } j \in \{ 1, \ldots, m-1\}$ (stückweise stetig)
    \item $g'(t_j \pm), g'(a+),g'(b-)$ (stückweise glatt)
\end{enumerate}
Sei $g$ stückweise glatt. Dann:
\begin{align*}
    g'(x_0) \coloneqq \frac{1}{2} (g'(x_0-) + g'(x_0+))
\end{align*}
\underline{Definition:} $\int \limits_{-\infty}^{\infty} f(t)\,dt$ heißt (absolut) konvergent $\Leftrightarrow \int \limits_{-\infty}^{\infty} \text{Re}f(t)\,dt$ 
\underline{und} $\int \limits_{-\infty}^{\infty} \text{Im}f(t)\,dt$ sind (absolut) konvergent. \\
Im Konvergenzfall: 
\begin{align*}
    \int \limits_{-\infty}^\infty f(t)\,dt \coloneqq  \int \limits_{-\infty}^\infty \cRe f(t)\,dt +  \ci \int \limits_{-\infty}^\infty \cIm f(t)\,dt
\end{align*} 
Ist $ \int \limits_{-\infty}^\infty f(t)\,dt$ absolut konvergent, so heißt $f$ \underline{absolut integrierbar (aib)}. \\
\underline{Fouriertransformierte:}
Zu einer Funktion $f: \mathbb{R} \to \mathbb{C}$ heißt $\widehat{f}:\mathbb{R} \to \mathbb{C}$ definiert durch:
\begin{align*}
    \widehat{f}(s) \coloneqq \frac{1}{2\pi} \int \limits_{-\infty}^\infty f(t) e^{-ist}\,dt
\end{align*}
die Fouriertransformierte von $f$. Die Zuordnung $f \mapsto \widehat{f}$ heißt \underline{Fouriertransformation}.

\subsection{Der Cauchysche Hauptwert}
Definition: $ \int \limits_{-\infty}^\infty f(x)\,dx= \lim \limits_{\beta \to - \infty} \int \limits_\beta^0 f(x)\,dx + \lim \limits_{\alpha \to \infty} \int \limits_0^\alpha f(x)\,dx$ \\
Existiert $\lim \limits_{\alpha \to \infty} \int \limits_{-\alpha}^\alpha f(x)\,dx$ so heißt diese Zahl \underline{Cauchyscher Hauptwert (CH)} und man schreibt: 
\begin{align*}
    \text{CH- } \int \limits_{- \infty}^\infty f(x)\,dx \coloneqq \lim \limits_{\alpha \to \infty} \int_{-\alpha}^\alpha f(x)\,dx
\end{align*}
Ist $\int \limits_{-\infty}^{\infty} f(x)\,dx$ konvergent, so ist $\text{CH-} \int \limits_{-\infty}^\infty f(x)\,dx \coloneqq \lim \limits_{\alpha \to \infty} \int \limits_{-\alpha}^\alpha f(x)\,dx$.
Sei $f: \mathbb{R} \to \mathbb{C}$ stückweise glatt und aib. Dann gilt:
\begin{align*}
    \forall t \in \mathbb{R}: \text{CH-}\int \limits_{-\infty}^\infty \widehat{f}(s)e^{ist} ds = \frac{1}{2} (f(t+) + f(t-))
\end{align*}
\underline{Nützliches zur Fouriertransformation:}
\begin{enumerate} [a)]
    \item $\widehat{\alpha f + \beta g} = \alpha \widehat{f} + \beta \widehat{g}$
    \item Es sei $V \coloneqq \{ f: \mathbb{R} \to \mathbb{C}: f \text{ ist stückweise stetig und absolut integrierbar}\}$. Für jedes $f \in V$ existiert die Fouriertransformierte
    \item Sei $f_h(t) \coloneqq f(t+h)$. Dann ist $f_h \in V$ und $\widehat{f_h}(s) = e^{ish} \widehat{f}(s) \text{ } (s \in \mathbb{R})$
    \item Sei $f: \mathbb{R} \to \mathbb{C}$ stückweise glatt, stetig und aib. Weiter sei $f'$ aib. Dann ist $f' \in V$ und $\widehat{f'}(s) = is\widehat{f}(s)$
\end{enumerate}

\subsection{Faltungen}
\underline{Definition:} Seien $f_1,f_2: \mathbb{R} \to \mathbb{C}$ Funktionen, sodass $\int \limits_{-\infty}^\infty f_1(t-x)f_2(x)\,dx$ für jedes $t \in \mathbb{R}$ konvergent ist. Dann heißt
\begin{align*}
    f_1 * f_2: \mathbb{R} \to \mathbb{C}, (f_1*f_2)(t) \coloneqq \frac{1}{2\pi} \int \limits_{-\infty}^\infty f_1(t-x)f_2(x)\,dx
\end{align*}
die \underline{Faltung} von $f_1$ und $f_2$. \\
\underline{Wichtiges zu Faltungen:} Seien $f_1,f_2: \mathbb{R} \to \mathbb{C}$ stetig und aib und $f_1$ beschränkt.
\begin{enumerate} [a)]
    \item $\forall t \in \mathbb{R}: \int \limits_{-\infty}^\infty f_1(t-x)f_2(x)\,dx$ konvergiert absolut
    \item $f_1*f_2$ ist stetig und aib und
    \begin{align*}
        (\widehat{f_1*f_2})(s) = \widehat{f_1}(s) \widehat{f_2}(s)
    \end{align*}
    \item $\lvert f_1*f_2(t)\rvert \leq \frac{1}{2\pi} \sup \limits_{x \in \mathbb{R}} \lvert f_1(x)\rvert \int \limits_{-\infty}^\infty \lvert f_2(x)\rvert\,dx \text{ } (t \in \mathbb{R})$
\end{enumerate}

\subsection{Schwartzraum}
Eine Funktion $f \in C^\infty(\mathbb{R},\mathbb{C})$ heißt \underline{schnell fallend} $\Leftrightarrow \forall m,n \in \mathbb{N}_0: t \mapsto t^mf^{(n)}(t)$ ist beschränkt auf $\mathbb{R}$
\begin{align*}
    S \coloneqq \{ f:\mathbb{R} \to \mathbb{C}: f \text{ ist schnell fallend}\}
\end{align*}
heißt \underline{Schwartz-Raum}. \\
\underline{Nützliches zu Funktionen des Schwartz-Raumes:}
Seien $f,g \in S$ und $p$ ein Polynom. Dann:
\begin{enumerate} [a)]
    \item $f$ ist aib und $\lim \limits_{t \to \pm \infty} f(t) = 0$
    \item $S$ ist ein Vektorraum $(\alpha f + \beta g \in S)$
    \item $fg,pf,\overline{f},\cRe f,\cIm f,t \mapsto f(-t) \in S$
    \item $\widehat{f} \in S$ und $f(t)= \int \limits_{-\infty}^\infty \widehat{f}(s)e^{ist}ds \text{ } (t \in \mathbb{R})$
    \item $f^{(n)} \in S \text{ } (n \in \mathbb{N})$ und $\widehat{f^{(n)}}(s) = (is)^n \widehat{f}(s)$
    \item $f*g \in S$ und $\widehat{f*g} = \widehat{f} \cdot \widehat{g}$
    \item Für $h(t) \coloneqq e^{-\frac{t^2}{2}} \text{ } (t \in \mathbb{R})$ gilt: $h \in S$ und $\widehat{h}= \frac{1}{\sqrt{2\pi}}h$ auf $\mathbb{R}$
\end{enumerate}
Die Fouriertransformation $f \mapsto \widehat{f}$ ist ein Isomorphismus von $S$ nach $S$!