\section{Differentialrechnung im $\mathbb{R}^n$ (reelwertige Funktionen)}
Definition:
\begin{enumerate}
    \item $f$ heißt in $x_0$ partiell differenzierbar (pdb) nach $x_i$: $\Leftrightarrow$
    Es existiert 
    \begin{align*}
        f_{x_i}(x_0) \coloneqq \frac{\partial f}{\partial x_i}(x_0) \coloneqq \lim \limits_{t \to 0} \frac{f(x_0 + te_i) - f(x_0)}{t} \in \mathbb{R}.
    \end{align*}
    In dem Fall heißt $f_{x_i}(x_0)$ die \underline{partielle Ableitung von $f$ in $x_0$ nach $x_i$.}
    \item $f$ heißt in $x_0 \in D$ pdb $\Leftrightarrow$ $f$ ist in $x_0$ pdb nach allen Variablen $x_1,\ldots,x_n$.
    I.d.F. heißt der Vektor
    \begin{align*}
        \text{grad} f(x_0) \coloneqq (f_{x_1}(x_0),\ldots,f_{x_n}(x_0))
    \end{align*}
    der \underline{Gradient von $f$ in $x_0$}
    \item Es sei $i \in \{1,\ldots,n\}$ und $f_{x_i}$ sei auf $D$ vorhanden. Also haben wir die partielle Ableitung von $f$ nach $x_i: f_{x_i}: D \to \mathbb{R}$.
    Es sei $x_0 \in D$ und $j \in \{1, \ldots,n \}$. Ist $f_{x_i}$ pdb nach $x_j$, so heißt
    \begin{align*}
        f_{x_ix_j}(x_0) \coloneqq \frac{\partial^2 f}{\partial x_j \partial x_i} (f_{x_i})_{x_j}(x_0)
    \end{align*}
    \underline{die partielle Ableitung 2. Ordnung von $f$ nach $x_i$ und $x_j$}
    \item $f \in \underline{C^m(D, \mathbb{R}}) :\Leftrightarrow$ $f$ ist auf $D$ $m$-mal stetig partiell differenzierbar 
    :$\Leftrightarrow$ alle partiellen Ableitungen von $f$ der Ordnung $\leq m$ sind auf $D$ vorhanden und dort stetig
\end{enumerate}

\subsection{Satz von Schwartz}
\label{sec: Schwartz}
Es sei $m \in \mathbb{N}$ und $f \in C^m(D,\mathbb{R})$. Dann ist jede partielle Ableitung von $f$ der Ordnung $\leq m$
unabhängig von der Reihenfolge der Differentiation.

\subsection{Differenzierbarkeit}
Definition: $f$ heißt \underline{in $x_0 \in D$ differenzierbar} (db): $\Leftrightarrow$
\begin{align*}
    \exists a \in \mathbb{R}^n: \lim \limits_{h \to 0} \frac{f(x_0+h) -f(x_0) - a\cdot h}{\lVert h \rVert} = 0 \\
    \Leftrightarrow \exists a \in \mathbb{R}^n : \lim \limits_{x \to x_0} \frac{f(x) -f(x_0) - a\cdot (x-x_0)}{\lVert x-x_0 \rVert} = 0
\end{align*}
\begin{enumerate}[a)]
    \item Ist $f$ in $x_0$ db, so ist $f$ in $x_0$ stetig und $f$ ist in $x_0$ pdb
    \item Ist $f$ in $x_0$ db, so ist der Vektor $a$ in obiger Definition eindeutig bestimmt und $a=\text{grad}f(x_0)$ I.d.F. gilt:
    \begin{align*}
        f'(x_0) \coloneqq a = \text{grad}f(x_0)
    \end{align*}
    \item Es sei $f$ auf $D$ pdb und $f_{x_1},\ldots,f_{x_n}$ seien in $x_0 \in D$ stetig. Dann ist $f$ in $x_0$ db.
\end{enumerate}

\subsection{Kettenregel}
Sei $I \subseteq \mathbb{R}$ ein Intervall, $g: I \to \mathbb{R}^n$ db in $f_0 \in I, g(I) \subseteq D$ und $f$ sei in $x_0 \coloneqq g(t_0)$ db. 
Dann ist $f \circ g: I \to \mathbb{R}$ db in $t_0$ und $(f \circ g)' (t_0) = f'(g(t_0)) \cdot g'(t_0)$, wobei hier das (euklidische) Skalarprodukt gemeint ist.

\subsection{Mittelwertsatz}
Es sei $f: D \to \mathbb{R}$ auf $D$ differenzierbar, es seien $a,b \in D$ und $S[a,b] \subseteq D$. Dann existiert ein $\xi \in S[a,b]$ mit $f(b) - f(a) = f'(\xi) \cdot (b-a)$

\subsection{Richtungen}
\begin{enumerate}[a)]
    \item Es sei $a \in \mathbb{R}^n$. Ist $\lVert a \rvert = 1,$ so heißt $a$ \underline{ein(e) Richtung(svektor)}
    \item Sei $x_0 \in D$ und $a \in \mathbb{R}^n$ eine Richtung. $f$ heißt \underline{in $x_0$ in Richtung $a$ differenzierbar} $\Leftrightarrow$ Es existiert der Grenzwert
    \begin{align*}
        \frac{\partial f}{\partial a}(x_0) \coloneqq \lim \limits_{t \to 0} \frac{f(x_0 + ta) -f(x_0)}{t} \in \mathbb{R}.
    \end{align*}
    I.d.F. heißt $\frac{\partial f}{\partial a}(x_0)$ die \underline{Richtungsableitung von $f$ in $x_0$ in Richtung $a$}.
    \item Ist $f$ in $x_0 \in D$ db und $a \in \mathbb{R}^n$ eine Richtung, so existiert $\frac{\partial f}{\partial a}(x_0)$ und 
    $\frac{\partial f}{\partial a}(x_0) = a \cdot \text{grad}f(x_0)$
\end{enumerate}

\subsection{Hesse-Matrix}
Es sei $f \in C^2(D,\mathbb{R})$ und $x_= \in D$. Die Matrix 
\begin{align*}
    H_f(x_0) \coloneqq \begin{pmatrix} f{x_1x_1}(x_0) & \ldots & f{x_1x_n}(x_0) \\ \vdots & & \vdots \\ f{x_nx_1}(x_0) & \ldots & f{x_nx_n}(x_0)\end{pmatrix}
\end{align*}
heißt \underline{Hesse-Matrix von $f$ in $x_0$}. Sie ist nach dem \hyperref[sec: Schwartz]{Satz von Schwartz} symmetrisch.

\subsection{Satz von Taylor}
Sei $f \in C^2(D,\mathbb{R}), x_0 \in D, h \in \mathbb{R}^n$ und $S[x_0,x_0+h] \subseteq D$. Dann existiert $\xi \in S[x_0,x_0+h]$ mit
\begin{align*}
    f(x_0+h) = f(x_0) + \text{ grad}f(x_0) \cdot h + \frac{1}{2} H_f(\xi) \cdot h
\end{align*}

\subsection{Definitheit}

Definition: Sei $A \in \mathbb{R}^{n \times n}$ symmetrisch. $A$ heißt
\begin{enumerate} [a)]
    \item positiv definit $\Leftrightarrow \forall x \in \mathbb{R}^n \setminus \{0\} (Ax) \cdot x > 0$
    \item negativ definit $ \Leftrightarrow \forall x \in \mathbb{R}^n \setminus \{0\} (Ax) \cdot x < 0$
    \item indefinit $\Leftrightarrow \exists u,v \in \mathbb{R}^n: (Au) \cdot u > 0$ und $(Av) \cdot v < 0$
\end{enumerate}
\vspace{0.5cm}

Kriterien für Definitheit:
\begin{enumerate} [a)]
    \item \begin{enumerate} [i)]
        \item $A$ ist positiv definit $\Leftrightarrow$ alle EW von $A$ sind $> 0$
        \item $A$ ist negativ definit $\Leftrightarrow$ alle EQ von $A$ sind $< 0$
        \item $A$ ist indefinit $\Leftrightarrow$ es gibt EW die $> 0$ und welche, die $< 0$ sind
    \end{enumerate}
    \item Sei $n=2, A = \begin{pmatrix} \alpha & \beta \\ \beta & \gamma\end{pmatrix}$
    \begin{enumerate} [i)]
        \item A positiv definit $\Leftrightarrow \alpha > 0, \text{ det } A > 0$
        \item A negativ definit $\Leftrightarrow \alpha < 0, \text{ det } A < 0$
        \item A ist indefinit $\Leftrightarrow \text{det } A < 0$
    \end{enumerate}
\end{enumerate}

\subsection{Lokale Extrema}
\begin{enumerate} [a)]
    \item Ist $f$ in $x_0 \in D$ pdb und hat $f$ in $x_0$ ein lokales Extremum, so ist $\text{grad}f(x_0) = 0.$
    \item Ist $f \in C^2(D,\mathbb{R}), x_0 \in D$ und $\text{grad}f(x_0) = 0$, so gilt:
    \begin{enumerate} [i)]
        \item Ist $H_f(x_0)$ positiv definit, so hat $f$ in $x_0$ ein lokales Minimum
        \item Ist $H_f(x_0)$ negativ definit, so hat $f$ in $x_0$ ein lokales Maximum
        \item Ist $H_f(x_0)$ indefinit, so hat $f$ in $x_0$ kein lokales Extremum
    \end{enumerate}
\end{enumerate}
\underline{Problem:} Überprüfe Menge auf Extrema
\underline{Lösung:} 
\begin{enumerate}
    \item Überprüfe, ob Menge überhaupt Extrempunkte haben kann/muss ($\widehat{=}$ dass sie kompakt ist)
    \item Überprüfe das innere der Menge auf Extrema
    \item Überprüfe Ränder der Menge auf Extram, sofern diese vorhanden sind
\end{enumerate}