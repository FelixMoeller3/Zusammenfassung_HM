\section{Fourierreihen}
Betrachte die Eigenschaft: $(V) \coloneqq f: \mathbb{R} \to \mathbb{R}, f \in R([-\pi,\pi])$ und $f$ ist \\
auf $\mathbb{R} \text{ }2\pi$-periodisch, d.h. $f(x+2\pi) = f(x) (x \in \mathbb{R})$ \\

Definition: Seien $(a_n)_{n=0}^{\infty}$ und $(b_n)_{n=0}^{\infty}$ Folgen in $\mathbb{R}$. Eine Reihe der Form
\begin{align*}
    \frac{a_0}{2} + \sum \limits_{n=1}^{\infty} (a_n \cos (nx) + b_n \sin (nx))
\end{align*}
heißt \underline{trigonometrische Reihe} (TR). 
\newpage 
\underline{Definition Fourierkoeffizienten/Fourierreihe}:
Die Funktion f erfülle $(V)$. Setze 
\begin{align*}
    a_n \coloneqq \frac{1}{\pi} \int_{-\pi}^{\pi} f(x) \cos(nx)\,dx, \hspace{0.5cm}b_n \coloneqq \frac{1}{\pi} \int_{-\pi}^{\pi} f(x) \sin(nx)\,dx
\end{align*}

Die Zahlen $a_n$ und $b_n$ heißen \underline{Fourierkoeffizienten (FK)} von $f$ und die mit $a_n$ und $b_n$ gebildete trigonometrische Reihe heißt die zu $f$
gehörende \underline{Fourierreihe}. Man schreibt: $f(x) \sim \frac{a_0}{2} + \sum \limits_{n=1}^{\infty} (a_n \cos(nx) + b_n \sin(nx))$

\subsection{Nützliches zu Fourierreihen und -koeffizienten}
Für $f$ gelte $(V)$.
\begin{enumerate} [a)]
    \item Ist $f$ gerade, also $f(x) = f(-x) (x \in \mathbb{R})$, so gilt für die Fourierkoeffizienten von f:
    \begin{align*}
        a_n = \frac{2}{\pi} \int_0^{\pi} f(x) \cos(nx)\,dx (n \in \mathbb{N}_0) \text{ und } b_n=0 (n \in \mathbb{N})
    \end{align*}
    \item Ist $f$ ungerade, also $f(x) = -f(-x) (x \in \mathbb{R}$, so gilt für die Fourierkoeffizienten von f:
    \begin{align*}
        a_n = 0 \text{ und } b_n= \frac{2}{\pi} \int_0^{\pi} f(x) \sin(nx)\,dx (n \in \mathbb{N}_0))
    \end{align*}
\end{enumerate}
Definition: Wir setzen $g(x_0\pm) \coloneqq \lim \limits_{x \to x_0 \pm} g(x)$, falls der Grenzwert existiert und reell ist.
Definition: Es sei $f:\mathbb{R} \to \mathbb{R} \text{ } 2\pi$-periodisch. Die Funktion heißt \underline{stückweise glatt}: $\Leftrightarrow$ es existiert eine
Zerlegung $\{t_0,t_1,\ldots,t_n\}$ des Intervalls $[-\pi,\pi]$ mit:
\begin{enumerate}[i)]
    \item $f \in C^1((t_{j-1},t_j)) (j=1,\ldots,n)$
    \item $\forall x \in \mathbb{R}: \exists f(x-),f'(x-),f(x+),f'(x+)$
\end{enumerate}
I.d.F. setzen wir $s_f(x) \coloneqq \frac{f(x+) + f(x-)}{2} (x \in \mathbb{R})$

\subsection{Konvergenz von Fourierreihen}
Die Funktion $f$ sei $2\pi$-periodisch und stückweise glatt. Dann konvergiert die Fourierreihe von $f$ in jedem $x \in \mathbb{R}$ gegen $s_f(x)$. Ist in diesem
Fall $f$ in $x \in \mathbb{R}$ stetig, so konvergiert die Fourierreihe gegen $f(x)$.

\subsection{Weitere Konvergenzerkenntnisse zu Fourierreihen}
Es sei $f \in C(\mathbb{R}), 2\pi$-periodisch und stückweise glatt. Dann gilt:
\begin{enumerate}
    \item Die Fourierreihe von $f$ konvergiert in jedem $x \in \mathbb{R}$ absolut und sie konvergiert auf $\mathbb{R}$ gleichmäßig gegen $f$.
    \item Die Reihen der Fourierkoeffizienten konvergieren absolut.
\end{enumerate}