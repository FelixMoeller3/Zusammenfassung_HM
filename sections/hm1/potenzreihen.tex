\section{Potenzreihen}
Konvergenzradius einer Potenzreihe (PR): $r \coloneqq \frac{1}{\limsup \limits_{n \to \infty} \sqrt[n]{|a_n|}}$, \\
wobei $\sum \limits_{n=0}^{\infty} a_n (x-x_0)^n$

\subsection{Quotientenkriterium für Potenzreihen}
Sei $a_n \neq 0$ ffa $n \in \mathbb{N}$. Wenn $|\frac{a_n}{a_{n+1}}|$ konvergiert, dann ist $r \coloneqq \lim \limits_{n \to \infty} |\frac{a_n}{a_{n+1}}|$

\subsection{Reihendarstellung von Sinus und Cosinus}
$\sin(x) \coloneqq \sum \limits_{n=0}^{\infty} (-1)^n \frac{x^{2n+1}}{(2n+1)!}$ \hspace{3em} $\cos(x) \coloneqq \sum \limits_{n=0}^{\infty} (-1)^n \frac{x^{2n}}{(2n)!}$ \\
Additionstheoreme: \\
$\sin(x+y) = \sin x \cos y + \sin y \cos x$ \\
$\cos(x+y) = \cos x \cos y - \sin x \sin y$