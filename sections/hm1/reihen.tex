\section{Reihen}
Wichtige Reihen:
\begin{enumerate}[a)]
    \item $\sum \limits_{n=1}^{\infty} x^n = \frac{1}{1-x}$ falls $|x| < 1$
    \item $\sum \limits_{n=1}^{\infty} \frac{1}{n(n+1)} = 1$
    \item $\sum \limits_{n=1}^{\infty} \frac{1}{n!} = e$
    \item $\sum \limits_{n=1}^{\infty} \frac{1}{n}$ ist divergent
    \item $\sum \limits_{n=1}^{\infty} \frac{(-1)^{n+1}}{n} = \log 2$
\end{enumerate}

\subsection{Monotoniekriterium}
\begin{enumerate} [a)]
    \item Sind alle $a_n \geq 0$ und ist $(s_n)$ beschränkt, so ist $\sum \limits_{n=1}^{\infty} a_n$ konvergent
    \item $\sum \limits_{n=1}^{\infty} a_n$ konvergiert $\Rightarrow a_n$ ist Nullfolge
\end{enumerate}

\subsection{Leibnizkriterium}
Sei $b_n$ eine monoton fallende Nullfolge. Dann konvergiert $\sum \limits_{n=1}^{\infty} (-1)^{n+1} b_n$

\subsection{Absolute Konvergenz}
Eine Reihe ist absolut konvergent genau dann, wenn $\sum \limits_{n=1}^{\infty} |b_n|$ konvergiert.
Absolute Konvergenz impliziert Konvergenz \\
Dreiecksungleichung für Reihen: $|\sum \limits_{n=1}^{\infty} a_n| \leq \sum \limits_{n=1}^{\infty} |a_n|$

\subsection{Majorantenkriterium}
Sei $|a_n| \leq b_n$ für fast alle $n \in \mathbb{N}$ und $\sum \limits_{n=1}^{\infty} b_n$ ist konvergent.
Dann ist $\sum \limits_{n=1}^{\infty} a_n$ absolut konvergent.

\subsection{Minorantenkriterium}
Sei $a_n \geq b_n > 0$ für fast alle $n \in \mathbb{N}$ und $\sum \limits_{n=1}^{\infty} b_n$ ist divergent.
Dann ist $\sum \limits_{n=1}^{\infty} a_n$ divergent.

\subsection{Wurzelkriterium}
Sei $(a_n)$ eine Folge. Dann gilt:
\begin{enumerate}[a)]
    \item $\sqrt[n]{|a_n|}$ ist unbeschränkt $\Rightarrow a_n$ ist divergent
    \item Ist $\limsup \limits_{n \to \infty} \sqrt[n]{|a_n|}
    \begin{cases} <1 \text{, so ist} \sum \limits_{n=1}^{\infty} a_n \text{ absolut konvergent} \\
    > 1 \text{, so ist}  \sum \limits_{n=1}^{\infty} a_n \text{ divergent} \end{cases}$ \\
    Im Fall $\limsup \limits_{n \to \infty} \sqrt[n]{|a_n|} = 1$ ist keine Aussage möglich
\end{enumerate}

\subsection{Quotientenkriterium}
Sei $a_n \neq 0$ ffa $n \in \mathbb{N}$ und $c_n \coloneqq |\frac{a_{n+1}}{a_n}|$. Dann gilt:
\begin{enumerate}[a)]
    \item $c_n \geq 1$ ffa $n \in \mathbb{N} \Rightarrow \sum \limits_{n=1}^{\infty} a_n$ ist divergent
    \item $\limsup \limits_{n \to \infty} c_n < 1 \Rightarrow \sum \limits_{n=1}^{\infty} a_n$ ist konvergent
    \item $\limsup \limits_{n \to \infty} c_n > 1 \Rightarrow \sum \limits_{n=1}^{\infty} a_n$ ist divergent
\end{enumerate}

\subsection{Cauchyprodukt}
Cauchyprodukt: $c_n \coloneqq \sum \limits_{k=0}^{n}b_{n-k}a_k$ \\
Seien $\sum \limits_{n=0}^{\infty} a_n$ und $\sum \limits_{n=0}^{\infty} b_n$ absolut konvergent. Dann gilt: \\
$\sum \limits_{n=0}^{\infty} c_n$ ist absolut konvergent und $\sum \limits_{n=0}^{\infty} c_n = (\sum \limits_{n=0}^{\infty} a_n) (\sum \limits_{n=0}^{\infty} b_n)$