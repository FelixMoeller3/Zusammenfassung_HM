\section{Uneigentliche Integrale}
Definition: Das uneigentliche Integral $\int_\alpha^\beta f(x)\,dx$ heißt konvergent $\Leftrightarrow$ 
\begin{align*}
    \text{Es existiert } \lim \limits_{t \to \beta -} \int_\alpha^t f(x)\,dx
\end{align*}
bzw.
\begin{align*}
    \lim \limits_{t \to \alpha +} \int_t^\beta f(x)\,dx
\end{align*}

Definition: Es sei $\alpha < \beta, \alpha \in \mathbb{R} \cup \{-\infty\}, \beta \in \mathbb{R} \cup \{\infty\}$ und 
$f:(a,b) \to \mathbb{R}$. Das uneigentliche Integral $\int_\alpha^\beta f(x)\,dx$ heißt konvergent: $\Leftrightarrow$
\begin{align*}
    \exists c \in (\alpha, \beta): \int_\alpha^c f(x)\,dx \text{ und } \int_c^\beta f(x)\,dx konvergieren
\end{align*}
Sonst ist $\int_\alpha^\beta f(x)\,dx$ divergent.

Definition $\int_\alpha^\beta f(x)\,dx$ heißt \underline{absolut konvergent} $\Leftrightarrow \int_\alpha^\beta |f(x)|\,dx$ konvergiert.

\subsection{Abschätzen von uneigentlichen Integralen}
\begin{enumerate} [a)]
    \item Ist $\int_\alpha^\beta f(x)\,dx$ absolut konvergent, so ist $\int_\alpha^\beta f(x)\,dx$ konvergent und 
    $|\int_\alpha^\beta f(x)\,dx| \leq \int_\alpha^\beta |f(x)|\,dx$
    \item \underline{Majorantenkriterium}: Ist $|f| \leq h$ auf $[\alpha,\beta)$ und $\int_\alpha^\beta h(x)\,dx$ konvergent, so ist
    $\int_\alpha^\beta f(x)\,dx$ absolut konvergent.
    \item \underline{Minorantenkriterium} Ist $f\geq h \geq 0$ auf $[a,b)$ und $\int_\alpha^\beta h(x)$ divergent, so ist 
    $\int_\alpha^\beta f(x)\,dx$ divergent.
\end{enumerate}