\section{Das Riemann-Integral}
Monotone und stetige Funktionen sind riemann-integrierbar.

\subsection{Erster Hauptsatz der Differential- und Integralrechnung}
Ist $f \in R([a,b])$ und besitzt eine Stammfunktion, so ist $\int_a^b f(x)dx = F(b) - F(a)$

\subsection{Integrale gleichmäßig konvergenter Funktionenfolgen}
Es sei $(f_n)$ eine Folge in $R([a,b])$ und $f_n$ konvergiert gleichmäßig auf $[a,b]$ gegen $f$. Dann:
\begin{enumerate}
    \item $f \in R([a,b])$
    \item $\lim \limits_{n \to \infty} \int_a^b f_n(x) dx = \int_a^b f(x) dx$
\end{enumerate}

\subsection{Integration von Potenzreihen}
$g(x) \coloneqq \sum \limits_{n=0}^{\infty} a_n (x-x_0)^n \Rightarrow G(x) \coloneqq \sum \limits_{n=0}^{\infty} \frac{a_n}{n+1} (x-x_0)^{n+1}$, 
wobei $G(x)$ den gleichen Konvergenzradius wie $g(x)$ besitzt

\subsection{Ein paar wichtige Erkenntnisse}
Es seien $f,g \in R(]a,b])$.
\begin{enumerate} [a)]
    \item Es sei $h$ lipschitzstetig auf $([a,b])$. Dann ist $h \circ f \in R([a,b])$
    \item $|f| \in R([a,b])$ und $|\int_a^b f(x) dx| \leq \int_a^b|f(x)| dx$ (Dreiecksungleichung für Integrale)
    \item $f \cdot g \in R([a,b])$ (Das Produkt integrierbarer Funktionen ist integrierbar)
    \item Ist $f(x) \neq 0 (x \in [a,b]$ und $\frac{1}{g}$ beschränkt auf $[a,b]$, so ist $\frac{1}{g}$ integrierbar.
\end{enumerate}

\subsection{Regeln für Integrationsgrenzen}
Es ist $\int_\alpha^\alpha f(x) dx \coloneqq 0$ und $\int_\alpha^\beta f(x) dx \coloneqq -\int_\beta^\alpha f(x) dx$

\subsection{Zweiter Hauptsatz}
Es sei $f \in R([a,b])$ und $F(x) \coloneqq \int_a^x f(t) dt (x \in [a,b])$. Dann gilt:
\begin{enumerate}[a)]
    \item $F(y) - F(x) = \int_x^y f(t) dt$
    \item $F$ ist Lipschitz-stetig
    \item Ist $f \in C([a,b])$, so ist $F \in C^1([a,b])$ und $F'(x) = f(x)$
\end{enumerate}
Folgerung: Stetige Funktionen (auf Intervallen) haben Stammfunktionen

\subsection{Partielle Integration}
\begin{align*}
    \int_a^b f'g dx = \Big[fg \Big]_{a}^b - \int_a^b fg' dx
\end{align*}

\subsection{Substitutionsregeln}
Voraussetzung: $I,J$ sind Intervalle in $\mathbb{R}$, $f \in C(I), g \in C^1(J)$ und $g(J) \subseteq I$.
\begin{enumerate}[a)]
    \item $\int f(g(t)) g'(t) dt = \int f(x) dx \mid_{x=g(t)}$
    \item Es sei $g(t) \neq 0$. Dann:
    \begin{align*}
        \int f(x) dx = \int f(g(t)) g'(t) dt \mid_{t=g^{-1}(x)}
    \end{align*}
    \item Ist $I = \langle a,b \rangle, J = \langle \alpha, \beta \rangle, g(\alpha) = a \text{und} g(\beta)=b$, so gilt
    \begin{align*}
        \int_a^b f(x) dx = \int_\alpha^\beta f(g(t))g'(t)dt
    \end{align*}
\end{enumerate}

\subsection{Mittelwertsatz der Integralrechnung}
Voraussetzung: $f,g \in R([a,b]), g \geq 0$ auf $[a,b], m \coloneqq \inf f([a,b])$ und $M \coloneqq \sup f([a,b])$. Dann gilt: \\
\begin{enumerate} [a)]
    \item $\exists \mu \in [m,M]: \int_a^b fg dx = \mu \int_a^b g dx$
    \item $\exists \mu \in [m.M]: \int_a^b f dx = \mu (b-a)$
\end{enumerate}
Ist $f \in C([a,b])$, so existiert ein $\xi \in [a,b]$ mit $\mu = f(\xi)$ in a) bzw. b)