\section{Funktionenfolgen und -reihen}
Die Funktionenfolge heißt auf $D$ \underline{punktweise konvergent} $\Leftrightarrow$ für jedes $x \in D$ ist die Folge $(f_n(x))$ konvergent.
I.d.F. sei $f(x) \coloneqq \lim \limits_{n \to \infty} f_n(x)$ (Grenzfunktion). 
Die Definition von \underline{Summenfunktionen} bei Funktionsreihen funktioniert analog.

\subsection{Gleichmäßige Konvergenz}
Definition: $\forall \varepsilon > 0: \exists n_0=n_0(\varepsilon) \in \mathbb{N}: \forall n \geq n_0: \forall x \in D; |f_n(x) - f(x)| < \varepsilon$ \\
Analog die Definition für Funktionenreihen

\subsection{Kriterien für gleichmäßige Konvergenz}
\begin{enumerate}[a)]
    \item $(f_n)$ konvergiere punktweise gegen $f$ und $(\alpha_n)$ sei eine Nullfolge. Dann gilt: \\
    $\forall n \geq m: \forall x \in D: |f_n(x) - f(x)| \leq \alpha_n \Rightarrow$ gleichmäßige Konvergenz
    \item Sei $(c_n)$ eine Folge in $[0,\infty)$, $\sum \limits_{n=1}^{\infty} c_n$ sei konvergent und $\forall n \geq m. \forall x \in D: |f_n(x)| \leq c_n$.
    Dann konvergiert $\sum \limits_{n=1}^{\infty} f_n$ auf $D$ gleichmäßig. (Kriterium von Weierstraß)
\end{enumerate}
Jede Potenzreihe konvergiert \textbf{in} $(x_0-r, x_0+r)$ gleichmäßig!

\subsection{Stetigkeit von Grenzfunktionen}
$(f_n)$ bzw. $\sum \limits_{n=1}^{\infty} f_n$ konvergiere gleichmäßig auf $D$ gegen $f$. Dann gilt: \\
\begin{enumerate}[a)]
    \item $\forall n \in \mathbb{N}: f_n$ ist stetig $\Rightarrow f$ ist stetig
    \item Sind alle $f_n \in C(D)$, so ist $f \in C(D)$
\end{enumerate}
Wichtige Folgerungen:
\begin{enumerate} [a)]
    \item Konvergiert $(f_n)$ punktweise gegen $f$ und gilt $f_n \in C(D) (n \in \mathbb{N})$ aber $f \notin C(D)$, so ist die Konvergenz \underline{nicht gleichmäßig}. 
    \item Wenn alle $f_n$ in $x_0$ stetig sind, dann gilt: 
    \begin{align*}
        \lim \limits_{x \to x_0} (\lim \limits_{n \to \infty} f_n(x)) =  \lim \limits_{n \to \infty} (\lim \limits_{x \to x_0} f_n(x))
    \end{align*}
\end{enumerate}