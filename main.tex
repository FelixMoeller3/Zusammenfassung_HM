\documentclass{article}
\usepackage[utf8]{inputenc}
\usepackage{enumerate}
\usepackage{amssymb}
\usepackage{amsmath}
\usepackage{mathtools}
\title{Höhere Mathematik für die Fachrichtung Informatik - Zusammenfassung}
\author{Felix Möller}
\date{August 2021}

\begin{document}

\maketitle

\section{Folgen}

\subsection{Monotoniekriterium}
Sei $(a_n)$ eine Folge. Dann gilt:
\begin{enumerate}[a)]
    \item $(a_n)$ ist monoton wachsend und nach oben beschränkt $\Rightarrow$ $(a_n)$ ist konvergent und
    $\lim \limits_{n \to \infty} a_n = \sup \limits_{n \in \mathbb{N}} a_n$
    \item $(a_n)$ ist monoton fallend und nach unten beschränkt $\Rightarrow$ $(a_n)$ ist konvergent und
    $\lim \limits_{n \to \infty} a_n = \inf \limits_{n \in \mathbb{N}} a_n$
\end{enumerate}

\subsection{Bolzano-Weierstraß}
Jede beschränkte Folge enthält eine konvergente Teilfolge

\section{Reihen}
Wichtige Reihen:
\begin{enumerate}[a)]
    \item $\sum \limits_{n=1}^{\infty} x^n = \frac{1}{1-x}$ falls $|x| < 1$
    \item $\sum \limits_{n=1}^{\infty} \frac{1}{n(n+1)} = 1$
    \item $\sum \limits_{n=1}^{\infty} \frac{1}{n!} = e$
    \item $\sum \limits_{n=1}^{\infty} \frac{1}{n}$ ist divergent
    \item $\sum \limits_{n=1}^{\infty} \frac{(-1)^{n+1}}{n} = \log 2$
\end{enumerate}

\subsection{Monotoniekriterium}
\begin{enumerate} [a)]
    \item Sind alle $a_n \geq 0$ und ist $(s_n)$ beschränkt, so ist $\sum \limits_{n=1}^{\infty} a_n$ konvergent
    \item $\sum \limits_{n=1}^{\infty} a_n$ konvergiert $\Rightarrow a_n$ ist Nullfolge
\end{enumerate}

\subsection{Leibnizkriterium}
Sei $b_n$ eine monoton fallende Nullfolge. Dann konvergiert $\sum \limits_{n=1}^{\infty} (-1)^{n+1} b_n$

\subsection{Absolute Konvergenz}
Eine Reihe ist absolut konvergent genau dann, wenn $\sum \limits_{n=1}^{\infty} |b_n|$ konvergiert.
Absolute Konvergenz impliziert Konvergenz \\
Dreiecksungleichung für Reihen: $|\sum \limits_{n=1}^{\infty} a_n| \leq \sum \limits_{n=1}^{\infty} |a_n|$

\subsection{Majorantenkriterium}
Sei $|a_n| \leq b_n$ für fast alle $n \in \mathbb{N}$ und $\sum \limits_{n=1}^{\infty} b_n$ ist konvergent.
Dann ist $\sum \limits_{n=1}^{\infty} a_n$ absolut konvergent.

\subsection{Minorantenkriterium}
Sei $a_n \geq b_n > 0$ für fast alle $n \in \mathbb{N}$ und $\sum \limits_{n=1}^{\infty} b_n$ ist divergent.
Dann ist $\sum \limits_{n=1}^{\infty} a_n$ divergent.

\subsection{Wurzelkriterium}
Sei $(a_n)$ eine Folge. Dann gilt:
\begin{enumerate}[a)]
    \item $\sqrt[n]{|a_n|}$ ist unbeschränkt $\Rightarrow a_n$ ist divergent
    \item Ist $\limsup \limits_{n \to \infty} \sqrt[n]{|a_n|}
    \begin{cases} <1 \text{, so ist} \sum \limits_{n=1}^{\infty} a_n \text{ absolut konvergent} \\
    > 1 \text{, so ist}  \sum \limits_{n=1}^{\infty} a_n \text{ divergent} \end{cases}$ \\
    Im Fall $\limsup \limits_{n \to \infty} \sqrt[n]{|a_n|} = 1$ ist keine Aussage möglich
\end{enumerate}

\subsection{Quotientenkriterium}
Sei $a_n \neq 0$ ffa $n \in \mathbb{N}$ und $c_n \coloneqq |\frac{a_{n+1}}{a_n}|$. Dann gilt:
\begin{enumerate}[a)]
    \item $c_n \geq 1$ ffa $n \in \mathbb{N} \Rightarrow \sum \limits_{n=1}^{\infty} a_n$ ist divergent
    \item $\limsup \limits_{n \to \infty} c_n < 1 \Rightarrow \sum \limits_{n=1}^{\infty} a_n$ ist konvergent
    \item $\limsup \limits_{n \to \infty} c_n > 1 \Rightarrow \sum \limits_{n=1}^{\infty} a_n$ ist divergent
\end{enumerate}

\subsection{Cauchyprodukt}
Cauchyprodukt: $c_n \coloneqq \sum \limits_{n=0}^{n}b_{n-k}a_k$ \\
Seien $\sum \limits_{n=0}^{\infty} a_n$ und $\sum \limits_{n=0}^{\infty} b_n$ absolut konvergent. Dann gilt: \\
$\sum \limits_{n=0}^{\infty} c_n$ ist absolut konvergent und $\sum \limits_{n=0}^{\infty} c_n = (\sum \limits_{n=0}^{\infty} a_n) (\sum \limits_{n=0}^{\infty} b_n)$

\section{Potenzreihen}
Konvergenzradius einer Potenzreihe (PR): $r \coloneqq \frac{1}{\limsup \limits_{n \to \infty} \sqrt[n]{|a_n|}}$, \\
wobei $\sum \limits_{n=0}^{\infty} a_n (x-x_0)^n$

\subsection{Quotientenkriterium für Potenzreihen}
Sei $a_n \neq 0$ ffa $n \in \mathbb{N}$. Wenn $|\frac{a_n}{a_{n+1}}|$ konvergiert, dann ist $r \coloneqq \lim \limits_{n \to \infty} |\frac{a_n}{a_{n+1}}|$

\subsection{Reihendarstellung von Sinus und Cosinus}
$\sin(x) \coloneqq \sum \limits_{n=0}^{\infty} (-1)^n \frac{x^{2n+1}}{(2n+1)!}$ \hspace{3em} $\cos(x) \coloneqq \sum \limits_{n=0}^{\infty} (-1)^n \frac{x^{2n}}{(2n)!}$ \\
Additionstheoreme: \\
$\sin(x+y) = \sin x \cos y + \sin y \cos x$ \\
$\cos(x+y) = \cos x \cos y - \sin x \sin y$

\section{q-adische Entwicklung}
Sei $x \in \mathbb{R}$. Verfahren der q-adischen Entwicklung einer Zahl $a$:
\begin{itemize}
    \item $z_0 \coloneqq [a]$
    \item $z_{n+1} \coloneqq [(a - z_0 - \frac{z_1}{q} - \ldots - \frac{z_n}{q^n}) q^{n+1}]$
\end{itemize}
q-adischer Bruch: $\sum \limits_{n=0}^{\infty} \frac{y_n}{q^n} = y_0,y_1y_2y_3\ldots$

\section{Stetigkeit}

\subsection{Zwischenwertsatz}
Es seien $a,b \in \mathbb{R}\text{, } a<b \text{, } f \in C ([a,b])$ und $y_0 \in [\text{min}\{f(a), f(b)\}, \text{max}\{f(a), f(b)\}]$. Dann existiert ein $x_0 \in [a,b] \text{mit} f(x_0)=y_0$

\subsection{Nullstellensatz von Bolzano}
Ist $f \in C([a,b])$ und $f(a)f(b) \leq 0$, so existiert ein $x_0 \in [a,b] \text{ mit } f(x_0)=0$ \\

\subsection{Abgeschlossenheit}
Für jede konvergente Folge $(x_n)$ in $D \subseteq \mathbb{R}$ gilt: $\lim \limits_{n \to \infty} x_n \in D$

\subsection{Kompaktheit}
Jede Folge $(x_n)$ in $D \subseteq \mathbb{R}$ enthält eine konvergente Teilfolge $(x_{n_k})$ mit $\lim \limits_{k \to \infty} x_{n_k} \in D$

\subsection{Logarithmusgesetze}
\begin{enumerate}[a)]
    \item $\forall x,y > 0: \log(xy) = \log x + \log y$
    \item $\forall x,y > 0: \log(\frac{x}{y}) = \log x - \log y$
\end{enumerate}

\subsection{Allgemeine Potenz}
$a^x \coloneqq e^{x \log a}$ \hspace{2em} $(a > 0)$

\subsection{Gleichmäßige Stetigkeit}
Sind $(x_n)(y_n)$ Folgen mit $x_n - y_n \to 0$, so gilt $f(x_n)-f(y_n) \to 0$

\subsection{Satz von Heine}
Jede stetige Funktion ist auf einem kompakte Intervall gleichmäßig stetig.

\subsection{Lipschitz-Stetigkeit}
$\exists L \geq 0: \forall x,y \in D: |f(x)-f(y)| \leq L|x-y|$ \\
f ist Lipschitz-stetig $\Rightarrow$ f ist gleichmäßig stetig

\section{Funktionenfolgen und -reihen}
Die Funktionenfolge heißt auf $D$ \underline{punktweise konvergent} $\Leftrightarrow$ für jedes $x \in D$ ist die Folge $(f_n(x))$ konvergent.
I.d.F. sei $f(x) \coloneqq \lim \limits_{n \to \infty} f_n(x)$ (Grenzfunktion). 
Die Definition von \underline{Summenfunktionen} bei Funktionsreihen funktioniert analog.

\subsection{Gleichmäßige Konvergenz}
Definition: $\forall \varepsilon > 0: \exists n_0=n_0(\varepsilon) \in \mathbb{N}: \forall n \geq n_0: \forall x \in D; |f_n(x) - f(x)| < \varepsilon$ \\
Analog die Definition für Funktionenreihen

\subsection{Kriterien für gleichmäßige Konvergenz}
\begin{enumerate}[a)]
    \item $(f_n)$ konvergiere punktweise gegen $f$ und $(\alpha_n)$ sei eine Nullfolge. Dann gilt: \\
    $\forall n \geq m: \forall x \in D: |f_n(x) - f(x)| \leq \alpha_n \Rightarrow$ gleichmäßige Konvergenz
    \item Sei $(c_n)$ eine Folge in $[0,\infty)$, $\sum \limits_{n=1}^{\infty} c_n$ sei konvergent und $\forall n \leq m. \forall x \in D: |f_n(x)| \leq c_n$.
    Dann konvergiert $\sum \limits_{n=1}^{\infty} f_n$ auf $D$ gleichmäßig. (Kriterium von Weierstraß)
\end{enumerate}
Jede Potenzreihe konvergiert \textbf{in} $(x_0-r, x_0+r)$ gleichmäßig!

\subsection{Stetigkeit von Grenzfunktionen}
$(f_n)$ bzw. $\sum \limits_{n=1}^{\infty} f_n$ konvergiere gleichmäßig auf $D$ gegen $f$. Dann gilt: \\
\begin{enumerate}[a)]
    \item $\forall n \in \mathbb{N}: f_n$ ist stetig $\Rightarrow f$ ist stetig
    \item Sind alle $f_n \in C(D)$, so ist $f \in C(D)$
\end{enumerate}
Wichtige Folgerungen:
\begin{enumerate} [a)]
    \item Konvergiert $(f_n)$ punktweise gegen $f$ und gilt $f_n \in C(D) (n \in \mathbb{N})$ aber $f \notin C(D)$, so ist die Konvergenz \underline{nicht gleichmäßig}. 
    \item Wenn alle $f_n$ in $x_0$ stetig sind, dann gilt: 
    \begin{align*}
        \lim \limits_{x \to x_0} (\lim \limits_{n \to \infty} f_n(x)) =  \lim \limits_{n \to \infty} (\lim \limits_{x \to x_0} f_n(x))
    \end{align*}
\end{enumerate}

\newpage


\section{Differentialrechnung}

\subsection{Differenzierbarkeitsbegriff}
Wenn $\lim \limits_{x \to x_0} \frac{f(x) - f(x_0)}{x - x_0}\Leftrightarrow \lim \limits_{h \to 0} \frac{f(x_0 + h) - f(x_0)}{h}$ existiert,
dann ist $f$ in $x_0$ \underline{differenzierbar}. Dazu gilt: $f$ ist differenzierbar $\Rightarrow$ $f$ ist stetig

\subsection{Differenzierbarkeitsregeln}
\begin{enumerate}[a)]
    \item $(\alpha f + \beta g)' = \alpha f' + \beta g'$ (Summenregel)
    \item $(fg)' = f'g + fg'$ (Produktregel)
    \item $(\frac{f}{g})' = \frac{f'g - fg'}{g^2}$ (Quotientenregel)
    \item $(f(g))' = f'(g) \cdot g'$ (Kettenregel)
\end{enumerate}

\subsection{Umkehrsatz}
Voraussetzung(en): $f$ ist stetig und streng monoton, in $x_0$ differenzierbar und $f'(x_0) \neq 0$. Dann gilt: \\
\begin{align*}
    (f^{-1)})'(y_0) = \frac{1}{f'(x_0)} = \frac{1}{f'(f^{-1}(y_0)}
\end{align*}

\subsection{Grenzwertdarstellung der eulerschen Zahl}
$e^{a} = \lim \limits_{x \to \infty} (1 + \frac{a}{x})^x$

\subsection{Ableitungswert an lokalen Extremstellen}
Voraussetzung: $x_0 \in I$ ist lokales Extremum von $f$, $f$ ist diffbar in $x_0$ und $x_0$ ist innerer Punkt von $I$. Dann gilt: $f'(x_0) = 0$.

\subsection{Mittelwertsatz}
Voraussetzung: $f \in C([a,b])$ und $f$ sei auf $(a,b)$ diffbar. Dann gilt: 
\begin{align*}
    \exists \xi \in (a,b): \frac{f(b) - f(a)}{b-a} = f'(\xi)
\end{align*}

\subsection{Satz von L'Hôpital}
Voraussetzung:
\begin{itemize}
    \item $I=(a,b)$, wobei $-\infty=a$ oder $b=\infty$ zulässig sind
    \item $f,g: I \to \mathbb{R}$ sind auf $I$ differenzierbar mit $g'(x) \neq 0 (x \in I)$
    \item $c=a \text{ oder } c=b$
\end{itemize}
Gilt $\lim \limits_{x \to c} f(x) = \lim \limits_{x \to c} g(x) = 0$ oder $\lim \limits_{x \to c} g(x) \pm \infty$, so ist 
\begin{align*}
    L \coloneqq \lim \limits_{x \to c} \frac{f'(x)}{g'(x)} =  \lim \limits_{x \to c} \frac{f(x)}{g(x)}
\end{align*}
für den Fall, dass $L$ existiert.

\subsection{Differenzieren von Potenzreihen}
Es sei $ f(x) \coloneqq \sum \limits_{n=0}^{\infty} a_n(x-x_0)^n$ eine Potenzreihe (PR) mit Konvergenzradius (KR) $r>0$. Dann gilt: \\
\begin{enumerate}[a)]
    \item Die PR $\sum \limits_{n=1}^{\infty} n \cdot a_n(x-x_0)^{n-1}$ hat KR $r$
    \item $f'(x) = \sum \limits_{n=1}^{\infty} n \cdot a_n(x-x_0)^{n-1}$
\end{enumerate}

\subsection{Tangens}
Die Funktion $\tan x \coloneqq \frac{\sin x}{\cos x}$ heißt Tangens. Es gilt: $(\tan x)' = \frac{1}{\cos^2(x)}$. \\
Die Umkehrfunktion des Tangens heißt Arkustangens. Es gilt: 
\begin{enumerate}[a)]
    \item $\arctan: \mathbb{R} \to (-\frac{\pi}{2},\frac{\pi}{2})$
    \item $(\arctan x)' = \frac{1}{1+x^2}$
\end{enumerate}

\subsection{Abelscher Grenzwertsatz und wichtige Reihenwerte}
Konvergiert eine PR $\sum \limits_{n=0}^{\infty} a_n(x-x_0)^n$ mit KR $r$ auch in $x_0 \pm r$, so ist sie auch stetig in $x_0 \pm r$. \\
Anwendung:
\begin{enumerate} [a)]
    \item $\log(1+x) = \sum \limits_{n=1}^{\infty} (-1)^{n+1} \frac{x^n}{n}$
    \item $\log 2 = \sum \limits_{n=1}^{\infty} \frac{(-1)^{n+1}}{n}$
    \item $\arctan x = \sum \limits_{n=0}^{\infty} (-1)^n \frac{x^{2n+1}}{2n+1}$
    \item $\arctan 1 = \frac{\pi}{4}$
\end{enumerate}

\newpage

\subsection{Satz von Taylor}
Voraussetzung: Es sei $n \in \mathbb{N}_0$ und $f$ auf $I$ $(n+1)$-mal differenzierbar. Dann existiert ein $\xi \in (\min\{x,x_0\}),\max\{x,x_0\}$ mit 
\begin{align*}
    f(x) = \sum \limits_{k=0}^{n} \frac{f^{(k)}(x_0)}{k!} (x-x_0)^k + \frac{f^{(n+1)} (\xi)}{(n+1)!} (x-x_0)^{n+1},
\end{align*}
wobei der erste Summand $T_n f(x,x_0) \coloneqq$ n-tes Taylorpolynom von $f$ im Punkt $x_0$ und der zweite Summand das Restglied ist.

\subsection{Bestimmung lokaler Extremstellen}
Es sei $n \geq 2 \text{, } f \in C^n(I) \text{, } x_0 \in I$ innerer Punkt von $I$, $f'(x_0) = f''(x_0)=\ldots=f^{(n-1)}(x_0)=0$ und $f^{(n)}(x_0) \neq 0$.
Dann gilt:
\begin{enumerate} [a)]
    \item Ist $n$ gerade und $f^{(n)}(x_0)<0$, so hat $f$ in $x_0$ ein lokales Maximum
    \item Ist $n$ gerade und $f^{(n)}(x_0)>0$, so hat $f$ in $x_0$ ein lokales Minimum
    \item Ist $n$ ungerade, so hat $f$ in $x_0$ kein lokales Extremum
\end{enumerate}

\section{Das Riemann-Integral}
Monotone und stetige Funktionen sind riemann-integrierbar.

\subsection{Erster Hauptsatz der Differential- und Integralrechnung}
Ist $f \in R([a,b])$ und besitzt eine Stammfunktion, so ist $\int_a^b f(x)dx = F(b) - F(a)$

\subsection{Integrale gleichmäßig konvergenter Funktionenfolgen}
Es sei $(f_n)$ eine Folge in $R([a,b])$ und $f_n$ konvergiert gleichmäßig auf $[a,b]$ gegen $f$. Dann:
\begin{enumerate}
    \item $f \in R([a,b])$
    \item $\lim \limits_{n \to \infty} \int_a^b f_n(x) dx = \int_a^b f(x) dx$
\end{enumerate}

\subsection{Integration von Potenzreihen}
$g(x) \coloneqq \sum \limits_{n=0}^{\infty} a_n (x-x_0)^n \Rightarrow G(x) \coloneqq \sum \limits_{n=0}^{\infty} \frac{a_n}{n+1} (x-x_0)^{n+1}$, 
wobei $G(x)$ den gleichen Konvergenzradius wie $g(x)$ besitzt

\subsection{Ein paar wichtige Erkenntnisse}
Es seien $f,g \in R(]a,b])$.
\begin{enumerate} [a)]
    \item Es sei $h$ lipschitzstetig auf $([a,b])$. Dann ist $h \circ f \in R([a,b])$
    \item $|f| \in R([a,b])$ und $|\int_a^b f(x) dx| \leq \int_a^b|f(x)| dx$ (Dreiecksungleichung für Integrale)
    \item $f \cdot g \in R([a,b])$ (Das Produkt integrierbarer Funktionen ist integrierbar)
    \item Ist $f(x) \neq 0 (x \in [a,b]$ und $\frac{1}{g}$ beschränkt auf $[a,b]$, so ist $\frac{1}{g}$ integrierbar.
\end{enumerate}

\subsection{Regeln für Integrationsgrenzen}
Es ist $\int_\alpha^\alpha f(x) dx \coloneqq 0$ und $\int_\alpha^\beta f(x) dx \coloneqq -\int_\beta^\alpha f(x) dx$

\subsection{Zweiter Hauptsatz}
Es sei $f \in R([a,b])$ und $F(x) \coloneqq \int_a^x f(t) dt (x \in [a,b])$. Dann gilt:
\begin{enumerate}[a)]
    \item $F(y) - F(x) = \int_x^y f(t) dt$
    \item $F$ ist Lipschitz-stetig
    \item Ist $f \in C([a,b])$, so ist $F \in C^1([a,b])$ und $F'(x) = f(x)$
\end{enumerate}
Folgerung: Stetige Funktionen (auf Intervallen) haben Stammfunktionen

\subsection{Partielle Integration}
\begin{align*}
    \int_a^b f'g dx = \Big[fg \Big]_{a}^b - \int_a^b fg' dx
\end{align*}

\subsection{Substitutionsregeln}
Voraussetzung: $I,J$ sind Intervalle in $\mathbb{R}$, $f \in C(I), g \in C^1(J)$ und $g(J) \subseteq I$.
\begin{enumerate}[a)]
    \item $\int f(g(t)) g'(t) dt = \int f(x) dx \mid_{x=g(t)}$
    \item Es sei $g(t) \neq 0$. Dann:
    \begin{align*}
        \int f(x) dx = \int f(g(t)) g'(t) dt \mid_{t=g^{-1}(x)}
    \end{align*}
    \item Ist $I = \langle a,b \rangle, J = \langle \alpha, \beta \rangle, g(\alpha) = a \text{und} g(\beta)=b$, so gilt
    \begin{align*}
        \int_a^b f(x) dx = \int_\alpha^\beta f(g(t))g'(t)dt
    \end{align*}
\end{enumerate}

\subsection{Mittelwertsatz der Integralrechnung}
Voraussetzung: $f,g \in R([a,b]), g \geq 0$ auf $[a,b], m \coloneqq \inf f([a,b])$ und $M \coloneqq \sup f([a,b])$. Dann gilt: \\
\begin{enumerate} [a)]
    \item $\exists \mu \in [m,M]: \int_a^b fg dx = \mu \int_a^b g dx$
    \item $\exists \mu \in [m.M]: \int_a^b f dx = \mu (b-a)$
\end{enumerate}
Ist $f \in C([a,b])$, so existiert ein $\xi \in [a,b]$ mit $\mu = f(\xi)$ in a) bzw. b)

\section{Uneigentliche Integrale}
Definition: Das uneigentliche Integral $\int_\alpha^\beta f(x)dx$ heißt konvergent $\Leftrightarrow$ 
\begin{align*}
    \text{Es existiert } \lim \limits_{t \to \beta -} \int_\alpha^t f(x)dx
\end{align*}
bzw.
\begin{align*}
    \lim \limits_{t \to \alpha +} \int_t^\beta f(x) dx
\end{align*}

Definition: Es sei $\alpha < \beta, \alpha \in \mathbb{R} \cup \{-\infty\}, \beta \in \mathbb{R} \cup \{\infty\}$ und 
$f:(a,b) \to \mathbb{R}$. Das uneigentliche Integral $\int_\alpha^\beta f(x) dx$ heißt konvergent: $\Leftrightarrow$
\begin{align*}
    \exists c \in (\alpha, \beta): \int_\alpha^c f(x)dx \text{ und } \int_c^\beta f(x)dx konvergieren
\end{align*}
Sonst ist $\int_\alpha^\beta f(x)dx$ divergent.

Definition $\int_\alpha^\beta f(x)dx$ heißt \underline{absolut konvergent} $\Leftrightarrow \int_\alpha^\beta |f(x)| dx$ konvergiert.

\subsection{Abschätzen von uneigentlichen Integralen}
\begin{enumerate} [a)]
    \item Ist $\int_\alpha^\beta f(x) dx$ absolut konvergent, so ist $\int_\alpha^\beta f(x) dx$ konvergent und 
    $|\int_\alpha^\beta f(x) dx| \leq \int_\alpha^\beta |f(x)| dx$
    \item \underline{Majorantenkriterium}: Ist $|f| \leq h$ auf $[\alpha,\beta)$ und $\int_\alpha^\beta h(x)dx$ konvergent, so ist
    $\int_\alpha^\beta f(x)dx$ absolut konvergent.
    \item \underline{Minorantenkriterium} Ist $f\geq h \geq 0$ auf $[a,b)$ und $\int_\alpha^\beta h(x)$ divergent, so ist 
    $\int_\alpha^\beta f(x)dx$ divergent.
\end{enumerate}

\section{Die Komplexe Exponentialfunktion}

\subsection{Einführung in komplexe Zahlen}
Die Menge $\mathbb{C}$ der komplexen Zahlen ist ein Körper. Die Binomische(n) Formel(n) gilt in $\mathbb{C}$ und die geometrische Summenformel
ebenfalls. Sei $z=x+iy \in \mathbb{C}$. Es gilt:
\begin{enumerate} [a)]
    \item $|z| \coloneqq \sqrt{x^2 + y^2}$ \underline{Betrag} von z
    \item $\Bar{z} \coloneqq x-iy$ ist das \underline{komplex Konjugierte} von $z=x+iy \in \mathbb{C}$
    \item $z \cdot \Bar{z} = |z|^2$
    \item $|z \cdot w| = |z| \cdot |w|$
    \item $|z+w| \leq |z| + |w|$
\end{enumerate}

Definition: Die auf $\mathbb{C}$ definierte Funktion
\begin{align*}
    z=x+iy \mapsto e^z \coloneqq e^x (\cos y) + i\sin y)
\end{align*}
heißt \underline{komplexe Exponentialfunktion}.

\subsection{Eigenschaften der komplexen Exponentialfunktion}
\begin{enumerate}
    \item $\forall z,w \in \mathbb{C}: e^{z+w} = e^z \cdot e^w, \forall z \in \mathbb{C} \forall n \in \mathbb{Z}: e^{nz}=(e^z)^n$
    \item $\forall t \in \mathbb{R}: |e^{it}| = 1$ und $e^{-it} = \overline{e^{it}}$
    \item $e^{\pi i} + 1 = 0$ (Eulersche Identität)
    \item $\forall k \in \mathbb{Z} \forall z \in \mathbb{C}: e^{z+2k\pi i} = e^z$
    \item $\forall z \in \mathbb{C}: \cos z = \frac{1}{2} (e^{iz} + e^{-iz})$ \, $\sin z = \frac{1}{2i}$
\end{enumerate}

\underline{Polarkoordinaten}: Sei $z=x+iy \in \mathbb{C} (x,y \in \mathbb{R}$ und $z \neq 0$. Setze $r \coloneqq |z| = \sqrt{x^2 + y^2}$.
Wähle Argument $\varphi$ von $z$: $\cos \varphi = \frac{x}{r}, \sin \varphi = \frac{y}{r}$. Also ist $z=x+iy = r\cos \varphi + i\sin \varphi = re^{i\varphi} = |z|e^{i \arg z}$

\subsection{Fundamentalsatz der Algebra}
Über $\mathbb{C}$ zerfällt jedes Polynom mit Grad $\geq 1$ in Linearfaktoren.

\subsection{Wurzeln in $\mathbb{C}$}
Die n-ten Einheitswurzeln einer Zahl $z$ aus $\mathbb{C}$ sind von der Form 
\begin{align*}
    z_k \coloneqq \sqrt[n]{r} e^{\frac{\varphi + 2k\pi}{n}} \text{ für ein } k \in \{0,1,\ldots,n-1\}
\end{align*}

\newpage

Möglichkeiten zum Wurzelziehen in $\mathbb{C}$:
\underline{Beispiel}: $\sqrt{-3+4i}$
\begin{enumerate}
    \item $w=u+iv$. Dann gilt
    \begin{align*}
        w^2 = u^2 + 2iuv - v^2 = -3+4i \Leftrightarrow u^2 - v^2 = -3 \text{ und } 2iuv=4i
    \end{align*}
    Löse das Gleichungssystem
    \item $z=-3+4i$. Bestimme $|z|$ und $\arg z$. Dann sind 
    \begin{align*}
        \pm \sqrt{|z|} e^{i \frac{\arg z}{2}} \text{ die Wurzeln von } z
    \end{align*}
    \item Ist $z \in (-\infty,0]$, so sind $w=\pm i \sqrt{-z}$ die Wurzeln von $z$
    \item pq-Formel
\end{enumerate}

\subsection{Komplexer Logarithmus}
Sei $w\in \mathbb{C} \setminus \{0\}, r=|w|$ und $\varphi = \arg w$. Für $z \in \mathbb{C}$ gilt: 
\begin{align*}
    z \text{ ist ein Logarithmus von } w \Leftrightarrow \exists k \in \mathbb{Z}: z = \log |w| + i\varphi + 2k\pi i
\end{align*}

\section{Fourierreihen}
Betrachte die Eigenschaft: $(V) \coloneqq f: \mathbb{R} \to \mathbb{R}, f \in R([-\pi,\pi])$ und $f$ ist \\
auf $\mathbb{R} \text{ }2\pi$-periodisch, d.h. $f(x+2\pi) = f(x) (x \in \mathbb{R})$ \\

Definition: Seien $(a_n)_{n=0}^{\infty}$ und $(b_n)_{n=0}^{\infty}$ Folgen in $\mathbb{R}$. Eine Reihe der Form
\begin{align*}
    \frac{a_0}{2} + \sum \limits_{n=1}^{\infty} (a_n \cos (nx) + b_n \sin (nx))
\end{align*}
heißt \underline{trigonometrische Reihe} (TR). \\

\underline{Definition Fourierkoeffizienten/Fourierreihe}:
Die Funktion f erfülle $(V)$. Setze 
\begin{align*}
    a_n \coloneqq \frac{1}{\pi} \int_{-\pi}^{\pi} f(x) \cos(nx) dx, \hspace{0.5cm}b_n \coloneqq \frac{1}{\pi} \int_{-\pi}^{\pi} f(x) \sin(nx) dx
\end{align*}

Die Zahlen $a_n$ und $b_n$ heißen \underline{Fourierkoeffizienten (FK)} von $f$ und die mit $a_n$ und $b_n$ gebildete trigonometrische Reihe heißt die zu $f$
gehörende \underline{Fourierreihe}. Man schreibt: $f(x) \sim \frac{a_0}{2} + \sum \limits_{n=1}^{\infty} (a_n \cos(nx) + b_n \sin(nx))$

\subsection{Nützliches zu Fourierreihen und -koeffizienten}
Für $f$ gelte $(V)$.
\begin{enumerate} [a)]
    \item Ist $f$ gerade, also $f(x) = f(-x) (x \in \mathbb{R})$, so gilt für die Fourierkoeffizienten von f:
    \begin{align*}
        a_n = \frac{2}{\pi} \int_0^{\pi} f(x) \cos(nx) dx (n \in \mathbb{N}_0) \text{ und } b_n=0 (n \in \mathbb{N})
    \end{align*}
    \item Ist $f$ ungerade, also $f(x) = -f(-x) (x \in \mathbb{R}$, so gilt für die Fourierkoeffizienten von f:
    \begin{align*}
        a_n = 0 \text{ und } b_n= \frac{2}{\pi} \int_0^{\pi} f(x) \sin(nx) dx (n \in \mathbb{N}_0))
    \end{align*}
\end{enumerate}
Definition: Wir setzen $g(x_0\pm) \coloneqq \lim \limits_{x \to x_0 \pm} g(x)$, falls der Grenzwert existiert und reell ist.
Definition: Es sei $f:\mathbb{R} \to \mathbb{R} \text{ } 2\pi$-periodisch. Die Funktion heißt \underline{stückweise glatt}: $\Leftrightarrow$ es existiert eine
Zerlegung $\{t_0,t_1,\ldots,t_n\}$ des Intervalls $[-\pi,\pi]$ mit:
\begin{enumerate}[i)]
    \item $f \in C^1((t_{j-1},t_j)) (j=1,\ldots,n)$
    \item $\forall x \in \mathbb{R}: \exists f(x-),f'(x-),f(x+),f'(x+)$
\end{enumerate}
I.d.F. setzen wir $s_f(x) \coloneqq \frac{f(x+) + f(x-)}{2} (x \in \mathbb{R})$

\subsection{Konvergenz von Fourierreihen}
Die Funktion $f$ sei $2\pi$-periodisch und stückweise glatt. Dann konvergiert die Fourierreihe von $f$ in jedem $x \in \mathbb{R}$ gegen $s_f(x)$. Ist in diesem
Fall $f$ in $x \in \mathbb{R}$ stetig, so konvergiert die Fourierreihe gegen $f(x)$.

\subsection{Weitere Konvergenzerkenntnisse zu Fourierreihen}
Es sei $f \in C(\mathbb{R}), 2\pi$-periodisch und stückweise glatt. Dann gilt:
\begin{enumerate}
    \item Die Fourierreihe von $f$ konvergiert in jedem $x \in \mathbb{R}$ absolut und sie konvergiert auf $\mathbb{R}$ gleichmäßig gegen $f$.
    \item Die Reihen der Fourierkoeffizienten konvergieren absolut.
\end{enumerate}

\section{Der Raum $\mathbb{R}^n$}
Definition: $xy \coloneqq x_1y_1 + \ldots + x_ny_n$ heißt \underline{Skalarprodukt}. Die Zahl $\| x \| \coloneqq \sqrt{x\cdot x} = \sqrt{x_1^2 + \ldots x_n^2}$
heißt Norm von $x$. Die Zahl $\| x-y\|$ heißt der Abstand von $x$ und $y$.

\subsection{Wichtige (Un)gleichungen}
Seien $x,y,z \in \mathbb{R}^n$ und $x \in \mathbb{R}$
\begin{enumerate} [a)]
    \item $(x+y) \cdot z = xz + yz$
    \item $\| \alpha x \| = |\alpha| \cdot \| x\|$
    \item $|x\cdot y| \leq \| x\| \cdot \| y\|$ (Cauchy-Schwarsche Ungleichung)
    \item $\|x+y\| \leq  \|x\| + \| y\|$ (Dreiecksungleichung)
    \item $|\| x \| - \| y \|| \leq \| x-y \|$
\end{enumerate}
\begin{align*}
    \|A\| = (\sum \limits_{j=1}^{m} \sum \limits_{k=1}^{n} a_{jk}^2)^{\frac{1}{2}} \text{ heißt die \underline{Norm von A}}
\end{align*}
Es gilt $\| AB\| \leq \| A \| \| B \|$ \\
Definition: $U_\varepsilon (x_0) \coloneqq \{x \in \mathbb{R}^n: |x-x_0| < \varepsilon \}$ heißt die \underline{offene Kugel um $x_0$} \\
$\overline{U_\varepsilon(x_0)} \coloneqq \{ x \in \mathbb{R}^n: |x-x_0| \leq \varepsilon\}$ heißt die \underline{abgeschlossene Kugel um $x_0$} \\

\underline{Definition}: Sei $A \subseteq \mathbb{R}^n$
\begin{enumerate} [a)]
    \item $A$ heißt \underline{beschränkt} $\Leftrightarrow \exists c \geq 0 \forall a \in A: \| a \| \leq c$
    \item $A$ heißt \underline{offen} $\Leftrightarrow \forall a \in A \exists \varepsilon = \varepsilon(a) > 0: U_\varepsilon(a) \subseteq A$ 
    \item $A$ heißt \underline{abgeschlossen} $\Leftrightarrow \mathbb{R}^n \setminus A$ ist offen
    \item $A$ heißt \underline{kompakt} $\Leftrightarrow A$ ist beschränkt und abgeschlossen 
\end{enumerate}
\end{document}
