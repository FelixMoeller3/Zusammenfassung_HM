\documentclass{article}
\usepackage[utf8]{inputenc}
\usepackage{enumerate}
\usepackage{amssymb}
\usepackage{amsmath}
\usepackage{mathtools}
\title{Höhere Mathematik für die Fachrichtung Informatik - Zusammenfassung}
\author{Felix Möller}
\date{August 2021}

\begin{document}

\maketitle

\section{Folgen}

\subsection{Monotoniekriterium}
Sei $(a_n)$ eine Folge. Dann gilt:
\begin{enumerate}[a)]
    \item $(a_n)$ ist monoton wachsend und nach oben beschränkt $\Rightarrow$ $(a_n)$ ist konvergent und
    $\lim \limits_{n \to \infty} a_n = \sup \limits_{n \in \mathbb{N}} a_n$
    \item $(a_n)$ ist monoton fallend und nach unten beschränkt $\Rightarrow$ $(a_n)$ ist konvergent und
    $\lim \limits_{n \to \infty} a_n = \inf \limits_{n \in \mathbb{N}} a_n$
\end{enumerate}

\subsection{Bolzano-Weierstraß}
Jede beschränkte Folge enthält eine konvergente Teilfolge

\section{Reihen}
Wichtige Reihen:
\begin{enumerate}[a)]
    \item $\sum \limits_{n=1}^{\infty} x^n = \frac{1}{1-x}$ falls $|x| < 1$
    \item $\sum \limits_{n=1}^{\infty} \frac{1}{n(n+1)} = 1$
    \item $\sum \limits_{n=1}^{\infty} \frac{1}{n!} = e$
    \item $\sum \limits_{n=1}^{\infty} \frac{1}{n}$ ist divergent
    \item $\sum \limits_{n=1}^{\infty} \frac{(-1)^{n+1}}{n} = \log 2$
\end{enumerate}

\subsection{Monotoniekriterium}
\begin{enumerate} [a)]
    \item Sind alle $a_n \geq 0$ und ist $(s_n)$ beschränkt, so ist $\sum \limits_{n=1}^{\infty} a_n$ konvergent
    \item $\sum \limits_{n=1}^{\infty} a_n$ konvergiert $\Rightarrow a_n$ ist Nullfolge
\end{enumerate}

\subsection{Leibnizkriterium}
Sei $b_n$ eine monoton fallende Nullfolge. Dann konvergiert $\sum \limits_{n=1}^{\infty} (-1)^{n+1} b_n$

\subsection{Absolute Konvergenz}
Eine Reihe ist absolut konvergent genau dann, wenn $\sum \limits_{n=1}^{\infty} |b_n|$ konvergiert.
Absolute Konvergenz impliziert Konvergenz \\
Dreiecksungleichung für Reihen: $|\sum \limits_{n=1}^{\infty} a_n| \leq \sum \limits_{n=1}^{\infty} |a_n|$

\subsection{Majorantenkriterium}
Sei $|a_n| \leq b_n$ für fast alle $n \in \mathbb{N}$ und $\sum \limits_{n=1}^{\infty} b_n$ ist konvergent.
Dann ist $\sum \limits_{n=1}^{\infty} a_n$ absolut konvergent.

\subsection{Minorantenkriterium}
Sei $a_n \geq b_n > 0$ für fast alle $n \in \mathbb{N}$ und $\sum \limits_{n=1}^{\infty} b_n$ ist divergent.
Dann ist $\sum \limits_{n=1}^{\infty} a_n$ divergent.

\subsection{Wurzelkriterium}
Sei $(a_n)$ eine Folge. Dann gilt:
\begin{enumerate}[a)]
    \item $\sqrt[n]{|a_n|}$ ist unbeschränkt $\Rightarrow a_n$ ist divergent
    \item Ist $\limsup \limits_{n \to \infty} \sqrt[n]{|a_n|}
    \begin{cases} <1 \text{, so ist} \sum \limits_{n=1}^{\infty} a_n \text{ absolut konvergent} \\
    > 1 \text{, so ist}  \sum \limits_{n=1}^{\infty} a_n \text{ divergent} \end{cases}$ \\
    Im Fall $\limsup \limits_{n \to \infty} \sqrt[n]{|a_n|} = 1$ ist keine Aussage möglich
\end{enumerate}

\subsection{Quotientenkriterium}
Sei $a_n \neq 0$ ffa $n \in \mathbb{N}$ und $c_n \coloneqq |\frac{a_{n+1}}{a_n}|$. Dann gilt:
\begin{enumerate}[a)]
    \item $c_n \geq 1$ ffa $n \in \mathbb{N} \Rightarrow \sum \limits_{n=1}^{\infty} a_n$ ist divergent
    \item $\limsup \limits_{n \to \infty} c_n < 1 \Rightarrow \sum \limits_{n=1}^{\infty} a_n$ ist konvergent
    \item $\limsup \limits_{n \to \infty} c_n > 1 \Rightarrow \sum \limits_{n=1}^{\infty} a_n$ ist divergent
\end{enumerate}

\subsection{Cauchyprodukt}
Cauchyprodukt: $c_n \coloneqq \sum \limits_{n=0}^{n}b_{n-k}a_k$ \\
Seien $\sum \limits_{n=0}^{\infty} a_n$ und $\sum \limits_{n=0}^{\infty} b_n$ absolut konvergent. Dann gilt: \\
$\sum \limits_{n=0}^{\infty} c_n$ ist absolut konvergent und $\sum \limits_{n=0}^{\infty} c_n = (\sum \limits_{n=0}^{\infty} a_n) (\sum \limits_{n=0}^{\infty} b_n)$

\section{Potenzreihen}
Konvergenzradius einer Potenzreihe (PR): $r \coloneqq \frac{1}{\limsup \limits_{n \to \infty} \sqrt[n]{|a_n|}}$, \\
wobei $\sum \limits_{n=0}^{\infty} a_n (x-x_0)^n$

\subsection{Quotientenkriterium für Potenzreihen}
Sei $a_n \neq 0$ ffa $n \in \mathbb{N}$. Wenn $|\frac{a_n}{a_{n+1}}|$ konvergiert, dann ist $r \coloneqq \lim \limits_{n \to \infty} |\frac{a_n}{a_{n+1}}|$

\subsection{Reihendarstellung von Sinus und Cosinus}
$\sin(x) \coloneqq \sum \limits_{n=0}^{\infty} (-1)^n \frac{x^{2n+1}}{(2n+1)!}$ \hspace{3em} $\cos(x) \coloneqq \sum \limits_{n=0}^{\infty} (-1)^n \frac{x^{2n}}{(2n)!}$ \\
Additionstheoreme: \\
$\sin(x+y) = \sin x \cos y + \sin y \cos x$ \\
$\cos(x+y) = \cos x \cos y - \sin x \sin y$

\section{q-adische Entwicklung}
Sei $x \in \mathbb{R}$. Verfahren der q-adischen Entwicklung einer Zahl $a$:
\begin{itemize}
    \item $z_0 \coloneqq [a]$
    \item $z_{n+1} \coloneqq [(a - z_0 - \frac{z_1}{q} - \ldots - \frac{z_n}{q^n}) q^{n+1}]$
\end{itemize}
q-adischer Bruch: $\sum \limits_{n=0}^{\infty} \frac{y_n}{q^n} = y_0,y_1y_2y_3\ldots$

\section{Stetigkeit}

\subsection{Zwischenwertsatz}
Es seien $a,b \in \mathbb{R}\text{, } a<b \text{, } f \in C ([a,b])$ und $y_0 \in [\text{min}\{f(a), f(b)\}, \text{max}\{f(a), f(b)\}]$. Dann existiert ein $x_0 \in [a,b] \text{mit} f(x_0)=y_0$

\subsection{Nullstellensatz von Bolzano}
Ist $f \in C([a,b])$ und $f(a)f(b) \leq 0$, so existiert ein $x_0 \in [a,b] \text{ mit } f(x_0)=0$ \\

\subsection{Abgeschlossenheit}
Für jede konvergente Folge $(x_n)$ in $D \subseteq \mathbb{R}$ gilt: $\lim \limits_{n \to \infty} x_n \in D$

\subsection{Kompaktheit}
Jede Folge $(x_n)$ in $D \subseteq \mathbb{R}$ enthält eine konvergente Teilfolge $(x_{n_k})$ mit $\lim \limits_{k \to \infty} x_{n_k} \in D$

\subsection{Logarithmusgesetze}
\begin{enumerate}[a)]
    \item $\forall x,y > 0: \log(xy) = \log x + \log y$
    \item $\forall x,y > 0: \log(\frac{x}{y}) = \log x - \log y$
\end{enumerate}

\subsection{Allgemeine Potenz}
$a^x \coloneqq e^{x \log a}$ \hspace{2em} $(a > 0)$

\subsection{Gleichmäßige Stetigkeit}
Sind $(x_n)(y_n)$ Folgen mit $x_n - y_n \to 0$, so gilt $f(x_n)-f(y_n) \to 0$

\subsection{Satz von Heine}
Jede stetige Funktion ist auf einem kompakte Intervall gleichmäßig stetig.

\subsection{Lipschitz-Stetigkeit}
$\exists L \geq 0: \forall x,y \in D: |f(x)-f(y)| \leq L|x-y|$ \\
f ist Lipschitz-stetig $\Rightarrow$ f ist gleichmäßig stetig

\section{Funktionenfolgen und -reihen}
Die Funktionenfolge heißt auf $D$ \underline{punktweise konvergent} $\Leftrightarrow$ für jedes $x \in D$ ist die Folge $(f_n(x))$ konvergent.
I.d.F. sei $f(x) \coloneqq \lim \limits_{n \to \infty} f_n(x)$ (Grenzfunktion). 
Die Definition von \underline{Summenfunktionen} bei Funktionsreihen funktioniert analog.

\subsection{Gleichmäßige Konvergenz}
Definition: $\forall \varepsilon > 0: \exists n_0=n_0(\varepsilon) \in \mathbb{N}: \forall n \geq n_0: \forall x \in D; |f_n(x) - f(x)| < \varepsilon$ \\
Analog die Definition für Funktionenreihen

\subsection{Kriterien für gleichmäßige Konvergenz}
\begin{enumerate}[a)]
    \item $(f_n)$ konvergiere punktweise gegen $f$ und $(\alpha_n)$ sei eine Nullfolge. Dann gilt: \\
    $\forall n \geq m: \forall x \in D: |f_n(x) - f(x)| \leq \alpha_n \Rightarrow$ gleichmäßige Konvergenz
    \item Sei $(c_n)$ eine Folge in $[0,\infty)$, $\sum \limits_{n=1}^{\infty} c_n$ sei konvergent und $\forall n \leq m. \forall x \in D: |f_n(x)| \leq c_n$.
    Dann konvergiert $\sum \limits_{n=1}^{\infty} f_n$ auf $D$ gleichmäßig. (Kriterium von Weierstraß)
\end{enumerate}
Jede Potenzreihe konvergiert \textbf{in} $(x_0-r, x_0+r)$ gleichmäßig!

\subsection{Stetigkeit von Grenzfunktionen}
$(f_n)$ bzw. $\sum \limits_{n=1}^{\infty} f_n$ konvergiere gleichmäßig auf $D$ gegen $f$. Dann gilt: \\
\begin{enumerate}[a)]
    \item $\forall n \in \mathbb{N}: f_n$ ist stetig $\Rightarrow f$ ist stetig
    \item Sind alle $f_n \in C(D)$, so ist $f \in C(D)$
\end{enumerate}
Wichtige Folgerungen:
\begin{enumerate} [a)]
    \item Konvergiert $(f_n)$ punktweise gegen $f$ und gilt $f_n \in C(D) (n \in \mathbb{N})$ aber $f \notin C(D)$, so ist die Konvergenz \underline{nicht gleichmäßig}. 
    \item Wenn alle $f_n$ in $x_0$ stetig sind, dann gilt: 
    \begin{align*}
        \lim \limits_{x \to x_0} (\lim \limits_{n \to \infty} f_n(x)) =  \lim \limits_{n \to \infty} (\lim \limits_{x \to x_0} f_n(x))
    \end{align*}
\end{enumerate}

\newpage


\section{Differentialrechnung}

\subsection{Differenzierbarkeitsbegriff}
Wenn $\lim \limits_{x \to x_0} \frac{f(x) - f(x_0)}{x - x_0}\Leftrightarrow \lim \limits_{h \to 0} \frac{f(x_0 + h) - f(x_0)}{h}$ existiert,
dann ist $f$ in $x_0$ \underline{differenzierbar}. Dazu gilt: $f$ ist differenzierbar $\Rightarrow$ $f$ ist stetig

\subsection{Differenzierbarkeitsregeln}
\begin{enumerate}[a)]
    \item $(\alpha f + \beta g)' = \alpha f' + \beta g'$ (Summenregel)
    \item $(fg)' = f'g + fg'$ (Produktregel)
    \item $(\frac{f}{g})' = \frac{f'g - fg'}{g^2}$ (Quotientenregel)
    \item $(f(g))' = f'(g) \cdot g'$ (Kettenregel)
\end{enumerate}

\subsection{Umkehrsatz}
Voraussetzung(en): $f$ ist stetig und streng monoton, in $x_0$ differenzierbar und $f'(x_0) \neq 0$. Dann gilt: \\
\begin{align*}
    (f^{-1)})'(y_0) = \frac{1}{f'(x_0)} = \frac{1}{f'(f^{-1}(y_0)}
\end{align*}

\subsection{Grenzwertdarstellung der eulerschen Zahl}
$e^{a} = \lim \limits_{x \to \infty} (1 + \frac{a}{x})^x$

\subsection{Ableitungswert an lokalen Extremstellen}
Voraussetzung: $x_0 \in I$ ist lokales Extremum von $f$, $f$ ist diffbar in $x_0$ und $x_0$ ist innerer Punkt von $I$. Dann gilt: $f'(x_0) = 0$.

\subsection{Mittelwertsatz}
Voraussetzung: $f \in C([a,b])$ und $f$ sei auf $(a,b)$ diffbar. Dann gilt: 
\begin{align*}
    \exists \xi
\end{align*}

\end{document}
