\documentclass{article}
\usepackage[utf8]{inputenc}
\usepackage{enumerate}
\usepackage{amssymb}
\usepackage{amsmath}
\usepackage{mathtools}
\usepackage{hyperref}
\title{Höhere Mathematik für die Fachrichtung Informatik - Zusammenfassung}
\author{Felix Möller}
\date{August 2021}

\newcommand{\ci}{\text{i}}
\newcommand{\cRe}{\text{Re}}
\newcommand{\cIm}{\text{Im}}

\begin{document}

\maketitle

\section{Folgen}

\subsection{Monotoniekriterium}
Sei $(a_n)$ eine Folge. Dann gilt:
\begin{enumerate}[a)]
    \item $(a_n)$ ist monoton wachsend und nach oben beschränkt $\Rightarrow$ $(a_n)$ ist konvergent und
    $\lim \limits_{n \to \infty} a_n = \sup \limits_{n \in \mathbb{N}} a_n$
    \item $(a_n)$ ist monoton fallend und nach unten beschränkt $\Rightarrow$ $(a_n)$ ist konvergent und
    $\lim \limits_{n \to \infty} a_n = \inf \limits_{n \in \mathbb{N}} a_n$
\end{enumerate}

\subsection{Bolzano-Weierstraß}
Jede beschränkte Folge enthält eine konvergente Teilfolge

\section{Reihen}
Wichtige Reihen:
\begin{enumerate}[a)]
    \item $\sum \limits_{n=1}^{\infty} x^n = \frac{1}{1-x}$ falls $|x| < 1$
    \item $\sum \limits_{n=1}^{\infty} \frac{1}{n(n+1)} = 1$
    \item $\sum \limits_{n=1}^{\infty} \frac{1}{n!} = e$
    \item $\sum \limits_{n=1}^{\infty} \frac{1}{n}$ ist divergent
    \item $\sum \limits_{n=1}^{\infty} \frac{(-1)^{n+1}}{n} = \log 2$
\end{enumerate}

\subsection{Monotoniekriterium}
\begin{enumerate} [a)]
    \item Sind alle $a_n \geq 0$ und ist $(s_n)$ beschränkt, so ist $\sum \limits_{n=1}^{\infty} a_n$ konvergent
    \item $\sum \limits_{n=1}^{\infty} a_n$ konvergiert $\Rightarrow a_n$ ist Nullfolge
\end{enumerate}

\subsection{Leibnizkriterium}
Sei $b_n$ eine monoton fallende Nullfolge. Dann konvergiert $\sum \limits_{n=1}^{\infty} (-1)^{n+1} b_n$

\subsection{Absolute Konvergenz}
Eine Reihe ist absolut konvergent genau dann, wenn $\sum \limits_{n=1}^{\infty} |b_n|$ konvergiert.
Absolute Konvergenz impliziert Konvergenz \\
Dreiecksungleichung für Reihen: $|\sum \limits_{n=1}^{\infty} a_n| \leq \sum \limits_{n=1}^{\infty} |a_n|$

\subsection{Majorantenkriterium}
Sei $|a_n| \leq b_n$ für fast alle $n \in \mathbb{N}$ und $\sum \limits_{n=1}^{\infty} b_n$ ist konvergent.
Dann ist $\sum \limits_{n=1}^{\infty} a_n$ absolut konvergent.

\subsection{Minorantenkriterium}
Sei $a_n \geq b_n > 0$ für fast alle $n \in \mathbb{N}$ und $\sum \limits_{n=1}^{\infty} b_n$ ist divergent.
Dann ist $\sum \limits_{n=1}^{\infty} a_n$ divergent.

\subsection{Wurzelkriterium}
Sei $(a_n)$ eine Folge. Dann gilt:
\begin{enumerate}[a)]
    \item $\sqrt[n]{|a_n|}$ ist unbeschränkt $\Rightarrow a_n$ ist divergent
    \item Ist $\limsup \limits_{n \to \infty} \sqrt[n]{|a_n|}
    \begin{cases} <1 \text{, so ist} \sum \limits_{n=1}^{\infty} a_n \text{ absolut konvergent} \\
    > 1 \text{, so ist}  \sum \limits_{n=1}^{\infty} a_n \text{ divergent} \end{cases}$ \\
    Im Fall $\limsup \limits_{n \to \infty} \sqrt[n]{|a_n|} = 1$ ist keine Aussage möglich
\end{enumerate}

\subsection{Quotientenkriterium}
Sei $a_n \neq 0$ ffa $n \in \mathbb{N}$ und $c_n \coloneqq |\frac{a_{n+1}}{a_n}|$. Dann gilt:
\begin{enumerate}[a)]
    \item $c_n \geq 1$ ffa $n \in \mathbb{N} \Rightarrow \sum \limits_{n=1}^{\infty} a_n$ ist divergent
    \item $\limsup \limits_{n \to \infty} c_n < 1 \Rightarrow \sum \limits_{n=1}^{\infty} a_n$ ist konvergent
    \item $\limsup \limits_{n \to \infty} c_n > 1 \Rightarrow \sum \limits_{n=1}^{\infty} a_n$ ist divergent
\end{enumerate}

\subsection{Cauchyprodukt}
Cauchyprodukt: $c_n \coloneqq \sum \limits_{n=0}^{n}b_{n-k}a_k$ \\
Seien $\sum \limits_{n=0}^{\infty} a_n$ und $\sum \limits_{n=0}^{\infty} b_n$ absolut konvergent. Dann gilt: \\
$\sum \limits_{n=0}^{\infty} c_n$ ist absolut konvergent und $\sum \limits_{n=0}^{\infty} c_n = (\sum \limits_{n=0}^{\infty} a_n) (\sum \limits_{n=0}^{\infty} b_n)$

\section{Potenzreihen}
Konvergenzradius einer Potenzreihe (PR): $r \coloneqq \frac{1}{\limsup \limits_{n \to \infty} \sqrt[n]{|a_n|}}$, \\
wobei $\sum \limits_{n=0}^{\infty} a_n (x-x_0)^n$

\subsection{Quotientenkriterium für Potenzreihen}
Sei $a_n \neq 0$ ffa $n \in \mathbb{N}$. Wenn $|\frac{a_n}{a_{n+1}}|$ konvergiert, dann ist $r \coloneqq \lim \limits_{n \to \infty} |\frac{a_n}{a_{n+1}}|$

\subsection{Reihendarstellung von Sinus und Cosinus}
$\sin(x) \coloneqq \sum \limits_{n=0}^{\infty} (-1)^n \frac{x^{2n+1}}{(2n+1)!}$ \hspace{3em} $\cos(x) \coloneqq \sum \limits_{n=0}^{\infty} (-1)^n \frac{x^{2n}}{(2n)!}$ \\
Additionstheoreme: \\
$\sin(x+y) = \sin x \cos y + \sin y \cos x$ \\
$\cos(x+y) = \cos x \cos y - \sin x \sin y$

\section{q-adische Entwicklung}
Sei $x \in \mathbb{R}$. Verfahren der q-adischen Entwicklung einer Zahl $a$:
\begin{itemize}
    \item $z_0 \coloneqq [a]$
    \item $z_{n+1} \coloneqq [(a - z_0 - \frac{z_1}{q} - \ldots - \frac{z_n}{q^n}) q^{n+1}]$
\end{itemize}
q-adischer Bruch: $\sum \limits_{n=0}^{\infty} \frac{y_n}{q^n} = y_0,y_1y_2y_3\ldots$

\section{Stetigkeit}

\subsection{Zwischenwertsatz}
Es seien $a,b \in \mathbb{R}\text{, } a<b \text{, } f \in C ([a,b])$ und $y_0 \in [\text{min}\{f(a), f(b)\}, \text{max}\{f(a), f(b)\}]$. Dann existiert ein $x_0 \in [a,b] \text{mit} f(x_0)=y_0$

\subsection{Nullstellensatz von Bolzano}
Ist $f \in C([a,b])$ und $f(a)f(b) \leq 0$, so existiert ein $x_0 \in [a,b] \text{ mit } f(x_0)=0$ \\

\subsection{Abgeschlossenheit}
Für jede konvergente Folge $(x_n)$ in $D \subseteq \mathbb{R}$ gilt: $\lim \limits_{n \to \infty} x_n \in D$

\subsection{Kompaktheit}
Jede Folge $(x_n)$ in $D \subseteq \mathbb{R}$ enthält eine konvergente Teilfolge $(x_{n_k})$ mit $\lim \limits_{k \to \infty} x_{n_k} \in D$

\subsection{Logarithmusgesetze}
\begin{enumerate}[a)]
    \item $\forall x,y > 0: \log(xy) = \log x + \log y$
    \item $\forall x,y > 0: \log(\frac{x}{y}) = \log x - \log y$
\end{enumerate}

\subsection{Allgemeine Potenz}
$a^x \coloneqq e^{x \log a}$ \hspace{2em} $(a > 0)$

\subsection{Gleichmäßige Stetigkeit}
Sind $(x_n)(y_n)$ Folgen mit $x_n - y_n \to 0$, so gilt $f(x_n)-f(y_n) \to 0$

\subsection{Satz von Heine}
Jede stetige Funktion ist auf einem kompakte Intervall gleichmäßig stetig.

\subsection{Lipschitz-Stetigkeit}
$\exists L \geq 0: \forall x,y \in D: |f(x)-f(y)| \leq L|x-y|$ \\
f ist Lipschitz-stetig $\Rightarrow$ f ist gleichmäßig stetig

\section{Funktionenfolgen und -reihen}
Die Funktionenfolge heißt auf $D$ \underline{punktweise konvergent} $\Leftrightarrow$ für jedes $x \in D$ ist die Folge $(f_n(x))$ konvergent.
I.d.F. sei $f(x) \coloneqq \lim \limits_{n \to \infty} f_n(x)$ (Grenzfunktion). 
Die Definition von \underline{Summenfunktionen} bei Funktionsreihen funktioniert analog.

\subsection{Gleichmäßige Konvergenz}
Definition: $\forall \varepsilon > 0: \exists n_0=n_0(\varepsilon) \in \mathbb{N}: \forall n \geq n_0: \forall x \in D; |f_n(x) - f(x)| < \varepsilon$ \\
Analog die Definition für Funktionenreihen

\subsection{Kriterien für gleichmäßige Konvergenz}
\begin{enumerate}[a)]
    \item $(f_n)$ konvergiere punktweise gegen $f$ und $(\alpha_n)$ sei eine Nullfolge. Dann gilt: \\
    $\forall n \geq m: \forall x \in D: |f_n(x) - f(x)| \leq \alpha_n \Rightarrow$ gleichmäßige Konvergenz
    \item Sei $(c_n)$ eine Folge in $[0,\infty)$, $\sum \limits_{n=1}^{\infty} c_n$ sei konvergent und $\forall n \leq m. \forall x \in D: |f_n(x)| \leq c_n$.
    Dann konvergiert $\sum \limits_{n=1}^{\infty} f_n$ auf $D$ gleichmäßig. (Kriterium von Weierstraß)
\end{enumerate}
Jede Potenzreihe konvergiert \textbf{in} $(x_0-r, x_0+r)$ gleichmäßig!

\subsection{Stetigkeit von Grenzfunktionen}
$(f_n)$ bzw. $\sum \limits_{n=1}^{\infty} f_n$ konvergiere gleichmäßig auf $D$ gegen $f$. Dann gilt: \\
\begin{enumerate}[a)]
    \item $\forall n \in \mathbb{N}: f_n$ ist stetig $\Rightarrow f$ ist stetig
    \item Sind alle $f_n \in C(D)$, so ist $f \in C(D)$
\end{enumerate}
Wichtige Folgerungen:
\begin{enumerate} [a)]
    \item Konvergiert $(f_n)$ punktweise gegen $f$ und gilt $f_n \in C(D) (n \in \mathbb{N})$ aber $f \notin C(D)$, so ist die Konvergenz \underline{nicht gleichmäßig}. 
    \item Wenn alle $f_n$ in $x_0$ stetig sind, dann gilt: 
    \begin{align*}
        \lim \limits_{x \to x_0} (\lim \limits_{n \to \infty} f_n(x)) =  \lim \limits_{n \to \infty} (\lim \limits_{x \to x_0} f_n(x))
    \end{align*}
\end{enumerate}

\newpage


\section{Differentialrechnung}

\subsection{Differenzierbarkeitsbegriff}
Wenn $\lim \limits_{x \to x_0} \frac{f(x) - f(x_0)}{x - x_0}\Leftrightarrow \lim \limits_{h \to 0} \frac{f(x_0 + h) - f(x_0)}{h}$ existiert,
dann ist $f$ in $x_0$ \underline{differenzierbar}. Dazu gilt: $f$ ist differenzierbar $\Rightarrow$ $f$ ist stetig

\subsection{Differenzierbarkeitsregeln}
\begin{enumerate}[a)]
    \item $(\alpha f + \beta g)' = \alpha f' + \beta g'$ (Summenregel)
    \item $(fg)' = f'g + fg'$ (Produktregel)
    \item $(\frac{f}{g})' = \frac{f'g - fg'}{g^2}$ (Quotientenregel)
    \item $(f(g))' = f'(g) \cdot g'$ (Kettenregel)
\end{enumerate}

\subsection{Umkehrsatz}
Voraussetzung(en): $f$ ist stetig und streng monoton, in $x_0$ differenzierbar und $f'(x_0) \neq 0$. Dann gilt: \\
\begin{align*}
    (f^{-1)})'(y_0) = \frac{1}{f'(x_0)} = \frac{1}{f'(f^{-1}(y_0)}
\end{align*}

\subsection{Grenzwertdarstellung der eulerschen Zahl}
$e^{a} = \lim \limits_{x \to \infty} (1 + \frac{a}{x})^x$

\subsection{Ableitungswert an lokalen Extremstellen}
Voraussetzung: $x_0 \in I$ ist lokales Extremum von $f$, $f$ ist diffbar in $x_0$ und $x_0$ ist innerer Punkt von $I$. Dann gilt: $f'(x_0) = 0$.

\subsection{Mittelwertsatz}
Voraussetzung: $f \in C([a,b])$ und $f$ sei auf $(a,b)$ diffbar. Dann gilt: 
\begin{align*}
    \exists \xi \in (a,b): \frac{f(b) - f(a)}{b-a} = f'(\xi)
\end{align*}

\subsection{Satz von L'Hôpital}
Voraussetzung:
\begin{itemize}
    \item $I=(a,b)$, wobei $-\infty=a$ oder $b=\infty$ zulässig sind
    \item $f,g: I \to \mathbb{R}$ sind auf $I$ differenzierbar mit $g'(x) \neq 0 (x \in I)$
    \item $c=a \text{ oder } c=b$
\end{itemize}
Gilt $\lim \limits_{x \to c} f(x) = \lim \limits_{x \to c} g(x) = 0$ oder $\lim \limits_{x \to c} g(x) \pm \infty$, so ist 
\begin{align*}
    L \coloneqq \lim \limits_{x \to c} \frac{f'(x)}{g'(x)} =  \lim \limits_{x \to c} \frac{f(x)}{g(x)}
\end{align*}
für den Fall, dass $L$ existiert.

\subsection{Differenzieren von Potenzreihen}
Es sei $ f(x) \coloneqq \sum \limits_{n=0}^{\infty} a_n(x-x_0)^n$ eine Potenzreihe (PR) mit Konvergenzradius (KR) $r>0$. Dann gilt: \\
\begin{enumerate}[a)]
    \item Die PR $\sum \limits_{n=1}^{\infty} n \cdot a_n(x-x_0)^{n-1}$ hat KR $r$
    \item $f'(x) = \sum \limits_{n=1}^{\infty} n \cdot a_n(x-x_0)^{n-1}$
\end{enumerate}

\subsection{Tangens}
Die Funktion $\tan x \coloneqq \frac{\sin x}{\cos x}$ heißt Tangens. Es gilt: $(\tan x)' = \frac{1}{\cos^2(x)}$. \\
Die Umkehrfunktion des Tangens heißt Arkustangens. Es gilt: 
\begin{enumerate}[a)]
    \item $\arctan: \mathbb{R} \to (-\frac{\pi}{2},\frac{\pi}{2})$
    \item $(\arctan x)' = \frac{1}{1+x^2}$
\end{enumerate}

\subsection{Abelscher Grenzwertsatz und wichtige Reihenwerte}
Konvergiert eine PR $\sum \limits_{n=0}^{\infty} a_n(x-x_0)^n$ mit KR $r$ auch in $x_0 \pm r$, so ist sie auch stetig in $x_0 \pm r$. \\
Anwendung:
\begin{enumerate} [a)]
    \item $\log(1+x) = \sum \limits_{n=1}^{\infty} (-1)^{n+1} \frac{x^n}{n}$
    \item $\log 2 = \sum \limits_{n=1}^{\infty} \frac{(-1)^{n+1}}{n}$
    \item $\arctan x = \sum \limits_{n=0}^{\infty} (-1)^n \frac{x^{2n+1}}{2n+1}$
    \item $\arctan 1 = \frac{\pi}{4}$
\end{enumerate}

\newpage

\subsection{Satz von Taylor}
Voraussetzung: Es sei $n \in \mathbb{N}_0$ und $f$ auf $I$ $(n+1)$-mal differenzierbar. Dann existiert ein $\xi \in (\min\{x,x_0\}),\max\{x,x_0\}$ mit 
\begin{align*}
    f(x) = \sum \limits_{k=0}^{n} \frac{f^{(k)}(x_0)}{k!} (x-x_0)^k + \frac{f^{(n+1)} (\xi)}{(n+1)!} (x-x_0)^{n+1},
\end{align*}
wobei der erste Summand $T_n f(x,x_0) \coloneqq$ n-tes Taylorpolynom von $f$ im Punkt $x_0$ und der zweite Summand das Restglied ist.

\subsection{Bestimmung lokaler Extremstellen}
Es sei $n \geq 2 \text{, } f \in C^n(I) \text{, } x_0 \in I$ innerer Punkt von $I$, $f'(x_0) = f''(x_0)=\ldots=f^{(n-1)}(x_0)=0$ und $f^{(n)}(x_0) \neq 0$.
Dann gilt:
\begin{enumerate} [a)]
    \item Ist $n$ gerade und $f^{(n)}(x_0)<0$, so hat $f$ in $x_0$ ein lokales Maximum
    \item Ist $n$ gerade und $f^{(n)}(x_0)>0$, so hat $f$ in $x_0$ ein lokales Minimum
    \item Ist $n$ ungerade, so hat $f$ in $x_0$ kein lokales Extremum
\end{enumerate}

\section{Das Riemann-Integral}
Monotone und stetige Funktionen sind riemann-integrierbar.

\subsection{Erster Hauptsatz der Differential- und Integralrechnung}
Ist $f \in R([a,b])$ und besitzt eine Stammfunktion, so ist $\int_a^b f(x)dx = F(b) - F(a)$

\subsection{Integrale gleichmäßig konvergenter Funktionenfolgen}
Es sei $(f_n)$ eine Folge in $R([a,b])$ und $f_n$ konvergiert gleichmäßig auf $[a,b]$ gegen $f$. Dann:
\begin{enumerate}
    \item $f \in R([a,b])$
    \item $\lim \limits_{n \to \infty} \int_a^b f_n(x) dx = \int_a^b f(x) dx$
\end{enumerate}

\subsection{Integration von Potenzreihen}
$g(x) \coloneqq \sum \limits_{n=0}^{\infty} a_n (x-x_0)^n \Rightarrow G(x) \coloneqq \sum \limits_{n=0}^{\infty} \frac{a_n}{n+1} (x-x_0)^{n+1}$, 
wobei $G(x)$ den gleichen Konvergenzradius wie $g(x)$ besitzt

\subsection{Ein paar wichtige Erkenntnisse}
Es seien $f,g \in R(]a,b])$.
\begin{enumerate} [a)]
    \item Es sei $h$ lipschitzstetig auf $([a,b])$. Dann ist $h \circ f \in R([a,b])$
    \item $|f| \in R([a,b])$ und $|\int_a^b f(x) dx| \leq \int_a^b|f(x)| dx$ (Dreiecksungleichung für Integrale)
    \item $f \cdot g \in R([a,b])$ (Das Produkt integrierbarer Funktionen ist integrierbar)
    \item Ist $f(x) \neq 0 (x \in [a,b]$ und $\frac{1}{g}$ beschränkt auf $[a,b]$, so ist $\frac{1}{g}$ integrierbar.
\end{enumerate}

\subsection{Regeln für Integrationsgrenzen}
Es ist $\int_\alpha^\alpha f(x) dx \coloneqq 0$ und $\int_\alpha^\beta f(x) dx \coloneqq -\int_\beta^\alpha f(x) dx$

\subsection{Zweiter Hauptsatz}
Es sei $f \in R([a,b])$ und $F(x) \coloneqq \int_a^x f(t) dt (x \in [a,b])$. Dann gilt:
\begin{enumerate}[a)]
    \item $F(y) - F(x) = \int_x^y f(t) dt$
    \item $F$ ist Lipschitz-stetig
    \item Ist $f \in C([a,b])$, so ist $F \in C^1([a,b])$ und $F'(x) = f(x)$
\end{enumerate}
Folgerung: Stetige Funktionen (auf Intervallen) haben Stammfunktionen

\subsection{Partielle Integration}
\begin{align*}
    \int_a^b f'g dx = \Big[fg \Big]_{a}^b - \int_a^b fg' dx
\end{align*}

\subsection{Substitutionsregeln}
Voraussetzung: $I,J$ sind Intervalle in $\mathbb{R}$, $f \in C(I), g \in C^1(J)$ und $g(J) \subseteq I$.
\begin{enumerate}[a)]
    \item $\int f(g(t)) g'(t) dt = \int f(x) dx \mid_{x=g(t)}$
    \item Es sei $g(t) \neq 0$. Dann:
    \begin{align*}
        \int f(x) dx = \int f(g(t)) g'(t) dt \mid_{t=g^{-1}(x)}
    \end{align*}
    \item Ist $I = \langle a,b \rangle, J = \langle \alpha, \beta \rangle, g(\alpha) = a \text{und} g(\beta)=b$, so gilt
    \begin{align*}
        \int_a^b f(x) dx = \int_\alpha^\beta f(g(t))g'(t)dt
    \end{align*}
\end{enumerate}

\subsection{Mittelwertsatz der Integralrechnung}
Voraussetzung: $f,g \in R([a,b]), g \geq 0$ auf $[a,b], m \coloneqq \inf f([a,b])$ und $M \coloneqq \sup f([a,b])$. Dann gilt: \\
\begin{enumerate} [a)]
    \item $\exists \mu \in [m,M]: \int_a^b fg dx = \mu \int_a^b g dx$
    \item $\exists \mu \in [m.M]: \int_a^b f dx = \mu (b-a)$
\end{enumerate}
Ist $f \in C([a,b])$, so existiert ein $\xi \in [a,b]$ mit $\mu = f(\xi)$ in a) bzw. b)

\section{Uneigentliche Integrale}
Definition: Das uneigentliche Integral $\int_\alpha^\beta f(x)dx$ heißt konvergent $\Leftrightarrow$ 
\begin{align*}
    \text{Es existiert } \lim \limits_{t \to \beta -} \int_\alpha^t f(x)dx
\end{align*}
bzw.
\begin{align*}
    \lim \limits_{t \to \alpha +} \int_t^\beta f(x) dx
\end{align*}

Definition: Es sei $\alpha < \beta, \alpha \in \mathbb{R} \cup \{-\infty\}, \beta \in \mathbb{R} \cup \{\infty\}$ und 
$f:(a,b) \to \mathbb{R}$. Das uneigentliche Integral $\int_\alpha^\beta f(x) dx$ heißt konvergent: $\Leftrightarrow$
\begin{align*}
    \exists c \in (\alpha, \beta): \int_\alpha^c f(x)dx \text{ und } \int_c^\beta f(x)dx konvergieren
\end{align*}
Sonst ist $\int_\alpha^\beta f(x)dx$ divergent.

Definition $\int_\alpha^\beta f(x)dx$ heißt \underline{absolut konvergent} $\Leftrightarrow \int_\alpha^\beta |f(x)| dx$ konvergiert.

\subsection{Abschätzen von uneigentlichen Integralen}
\begin{enumerate} [a)]
    \item Ist $\int_\alpha^\beta f(x) dx$ absolut konvergent, so ist $\int_\alpha^\beta f(x) dx$ konvergent und 
    $|\int_\alpha^\beta f(x) dx| \leq \int_\alpha^\beta |f(x)| dx$
    \item \underline{Majorantenkriterium}: Ist $|f| \leq h$ auf $[\alpha,\beta)$ und $\int_\alpha^\beta h(x)dx$ konvergent, so ist
    $\int_\alpha^\beta f(x)dx$ absolut konvergent.
    \item \underline{Minorantenkriterium} Ist $f\geq h \geq 0$ auf $[a,b)$ und $\int_\alpha^\beta h(x)$ divergent, so ist 
    $\int_\alpha^\beta f(x)dx$ divergent.
\end{enumerate}

\section{Die Komplexe Exponentialfunktion}

\subsection{Einführung in komplexe Zahlen}
Die Menge $\mathbb{C}$ der komplexen Zahlen ist ein Körper. Die Binomische(n) Formel(n) gilt in $\mathbb{C}$ und die geometrische Summenformel
ebenfalls. Sei $z=x+iy \in \mathbb{C}$. Es gilt:
\begin{enumerate} [a)]
    \item $|z| \coloneqq \sqrt{x^2 + y^2}$ \underline{Betrag} von z
    \item $\Bar{z} \coloneqq x-iy$ ist das \underline{komplex Konjugierte} von $z=x+iy \in \mathbb{C}$
    \item $z \cdot \Bar{z} = |z|^2$
    \item $|z \cdot w| = |z| \cdot |w|$
    \item $|z+w| \leq |z| + |w|$
\end{enumerate}

Definition: Die auf $\mathbb{C}$ definierte Funktion
\begin{align*}
    z=x+iy \mapsto e^z \coloneqq e^x (\cos y) + i\sin y)
\end{align*}
heißt \underline{komplexe Exponentialfunktion}.

\subsection{Eigenschaften der komplexen Exponentialfunktion}
\begin{enumerate}
    \item $\forall z,w \in \mathbb{C}: e^{z+w} = e^z \cdot e^w, \forall z \in \mathbb{C} \forall n \in \mathbb{Z}: e^{nz}=(e^z)^n$
    \item $\forall t \in \mathbb{R}: |e^{it}| = 1$ und $e^{-it} = \overline{e^{it}}$
    \item $e^{\pi i} + 1 = 0$ (Eulersche Identität)
    \item $\forall k \in \mathbb{Z} \forall z \in \mathbb{C}: e^{z+2k\pi i} = e^z$
    \item $\forall z \in \mathbb{C}: \cos z = \frac{1}{2} (e^{iz} + e^{-iz})$ \, $\sin z = \frac{1}{2i}$
\end{enumerate}

\underline{Polarkoordinaten}: Sei $z=x+iy \in \mathbb{C} (x,y \in \mathbb{R}$ und $z \neq 0$. Setze $r \coloneqq |z| = \sqrt{x^2 + y^2}$.
Wähle Argument $\varphi$ von $z$: $\cos \varphi = \frac{x}{r}, \sin \varphi = \frac{y}{r}$. Also ist $z=x+iy = r\cos \varphi + i\sin \varphi = re^{i\varphi} = |z|e^{i \arg z}$

\subsection{Fundamentalsatz der Algebra}
Über $\mathbb{C}$ zerfällt jedes Polynom mit Grad $\geq 1$ in Linearfaktoren.

\subsection{Wurzeln in $\mathbb{C}$}
Die n-ten Einheitswurzeln einer Zahl $z$ aus $\mathbb{C}$ sind von der Form 
\begin{align*}
    z_k \coloneqq \sqrt[n]{r} e^{\frac{\varphi + 2k\pi}{n}} \text{ für ein } k \in \{0,1,\ldots,n-1\}
\end{align*}

\newpage

Möglichkeiten zum Wurzelziehen in $\mathbb{C}$:
\underline{Beispiel}: $\sqrt{-3+4i}$
\begin{enumerate}
    \item $w=u+iv$. Dann gilt
    \begin{align*}
        w^2 = u^2 + 2iuv - v^2 = -3+4i \Leftrightarrow u^2 - v^2 = -3 \text{ und } 2iuv=4i
    \end{align*}
    Löse das Gleichungssystem
    \item $z=-3+4i$. Bestimme $|z|$ und $\arg z$. Dann sind 
    \begin{align*}
        \pm \sqrt{|z|} e^{i \frac{\arg z}{2}} \text{ die Wurzeln von } z
    \end{align*}
    \item Ist $z \in (-\infty,0]$, so sind $w=\pm i \sqrt{-z}$ die Wurzeln von $z$
    \item pq-Formel
\end{enumerate}

\subsection{Komplexer Logarithmus}
Sei $w\in \mathbb{C} \setminus \{0\}, r=|w|$ und $\varphi = \arg w$. Für $z \in \mathbb{C}$ gilt: 
\begin{align*}
    z \text{ ist ein Logarithmus von } w \Leftrightarrow \exists k \in \mathbb{Z}: z = \log |w| + i\varphi + 2k\pi i
\end{align*}

\section{Fourierreihen}
Betrachte die Eigenschaft: $(V) \coloneqq f: \mathbb{R} \to \mathbb{R}, f \in R([-\pi,\pi])$ und $f$ ist \\
auf $\mathbb{R} \text{ }2\pi$-periodisch, d.h. $f(x+2\pi) = f(x) (x \in \mathbb{R})$ \\

Definition: Seien $(a_n)_{n=0}^{\infty}$ und $(b_n)_{n=0}^{\infty}$ Folgen in $\mathbb{R}$. Eine Reihe der Form
\begin{align*}
    \frac{a_0}{2} + \sum \limits_{n=1}^{\infty} (a_n \cos (nx) + b_n \sin (nx))
\end{align*}
heißt \underline{trigonometrische Reihe} (TR). \\

\underline{Definition Fourierkoeffizienten/Fourierreihe}:
Die Funktion f erfülle $(V)$. Setze 
\begin{align*}
    a_n \coloneqq \frac{1}{\pi} \int_{-\pi}^{\pi} f(x) \cos(nx) dx, \hspace{0.5cm}b_n \coloneqq \frac{1}{\pi} \int_{-\pi}^{\pi} f(x) \sin(nx) dx
\end{align*}

Die Zahlen $a_n$ und $b_n$ heißen \underline{Fourierkoeffizienten (FK)} von $f$ und die mit $a_n$ und $b_n$ gebildete trigonometrische Reihe heißt die zu $f$
gehörende \underline{Fourierreihe}. Man schreibt: $f(x) \sim \frac{a_0}{2} + \sum \limits_{n=1}^{\infty} (a_n \cos(nx) + b_n \sin(nx))$

\subsection{Nützliches zu Fourierreihen und -koeffizienten}
Für $f$ gelte $(V)$.
\begin{enumerate} [a)]
    \item Ist $f$ gerade, also $f(x) = f(-x) (x \in \mathbb{R})$, so gilt für die Fourierkoeffizienten von f:
    \begin{align*}
        a_n = \frac{2}{\pi} \int_0^{\pi} f(x) \cos(nx) dx (n \in \mathbb{N}_0) \text{ und } b_n=0 (n \in \mathbb{N})
    \end{align*}
    \item Ist $f$ ungerade, also $f(x) = -f(-x) (x \in \mathbb{R}$, so gilt für die Fourierkoeffizienten von f:
    \begin{align*}
        a_n = 0 \text{ und } b_n= \frac{2}{\pi} \int_0^{\pi} f(x) \sin(nx) dx (n \in \mathbb{N}_0))
    \end{align*}
\end{enumerate}
Definition: Wir setzen $g(x_0\pm) \coloneqq \lim \limits_{x \to x_0 \pm} g(x)$, falls der Grenzwert existiert und reell ist.
Definition: Es sei $f:\mathbb{R} \to \mathbb{R} \text{ } 2\pi$-periodisch. Die Funktion heißt \underline{stückweise glatt}: $\Leftrightarrow$ es existiert eine
Zerlegung $\{t_0,t_1,\ldots,t_n\}$ des Intervalls $[-\pi,\pi]$ mit:
\begin{enumerate}[i)]
    \item $f \in C^1((t_{j-1},t_j)) (j=1,\ldots,n)$
    \item $\forall x \in \mathbb{R}: \exists f(x-),f'(x-),f(x+),f'(x+)$
\end{enumerate}
I.d.F. setzen wir $s_f(x) \coloneqq \frac{f(x+) + f(x-)}{2} (x \in \mathbb{R})$

\subsection{Konvergenz von Fourierreihen}
Die Funktion $f$ sei $2\pi$-periodisch und stückweise glatt. Dann konvergiert die Fourierreihe von $f$ in jedem $x \in \mathbb{R}$ gegen $s_f(x)$. Ist in diesem
Fall $f$ in $x \in \mathbb{R}$ stetig, so konvergiert die Fourierreihe gegen $f(x)$.

\subsection{Weitere Konvergenzerkenntnisse zu Fourierreihen}
Es sei $f \in C(\mathbb{R}), 2\pi$-periodisch und stückweise glatt. Dann gilt:
\begin{enumerate}
    \item Die Fourierreihe von $f$ konvergiert in jedem $x \in \mathbb{R}$ absolut und sie konvergiert auf $\mathbb{R}$ gleichmäßig gegen $f$.
    \item Die Reihen der Fourierkoeffizienten konvergieren absolut.
\end{enumerate}

\section{Der Raum $\mathbb{R}^n$}
Definition: $xy \coloneqq x_1y_1 + \ldots + x_ny_n$ heißt \underline{Skalarprodukt}. Die Zahl $\| x \| \coloneqq \sqrt{x\cdot x} = \sqrt{x_1^2 + \ldots x_n^2}$
heißt Norm von $x$. Die Zahl $\| x-y\|$ heißt der Abstand von $x$ und $y$.

\subsection{Wichtige (Un)gleichungen}
Seien $x,y,z \in \mathbb{R}^n$ und $x \in \mathbb{R}$
\begin{enumerate} [a)]
    \item $(x+y) \cdot z = xz + yz$
    \item $\| \alpha x \| = |\alpha| \cdot \| x\|$
    \item $|x\cdot y| \leq \| x\| \cdot \| y\|$ (Cauchy-Schwarsche Ungleichung)
    \item $\|x+y\| \leq  \|x\| + \| y\|$ (Dreiecksungleichung)
    \item $|\| x \| - \| y \|| \leq \| x-y \|$
\end{enumerate}
\begin{align*}
    \|A\| = (\sum \limits_{j=1}^{m} \sum \limits_{k=1}^{n} a_{jk}^2)^{\frac{1}{2}} \text{ heißt die \underline{Norm von A}}
\end{align*}
Es gilt $\| AB\| \leq \| A \| \| B \|$ \\
Definition: $U_\varepsilon (x_0) \coloneqq \{x \in \mathbb{R}^n: |x-x_0| < \varepsilon \}$ heißt die \underline{offene Kugel um $x_0$} \\
$\overline{U_\varepsilon(x_0)} \coloneqq \{ x \in \mathbb{R}^n: |x-x_0| \leq \varepsilon\}$ heißt die \underline{abgeschlossene Kugel um $x_0$} \\

\underline{Definition}: Sei $A \subseteq \mathbb{R}^n$
\begin{enumerate} [a)]
    \item $A$ heißt \underline{beschränkt} $\Leftrightarrow \exists c \geq 0 \forall a \in A: \| a \| \leq c$
    \item $A$ heißt \underline{offen} $\Leftrightarrow \forall a \in A \exists \varepsilon = \varepsilon(a) > 0: U_\varepsilon(a) \subseteq A$ 
    \item $A$ heißt \underline{abgeschlossen} $\Leftrightarrow \mathbb{R}^n \setminus A$ ist offen
    \item $A$ heißt \underline{kompakt} $\Leftrightarrow A$ ist beschränkt und abgeschlossen 
\end{enumerate}

\section{Konvergenz im $\mathbb{R}^n$}
Eine Folge $(a^{(k)})$ heißt konvergent $\Leftrightarrow \exists a \in \mathbb{R}^n: \lVert a^{(k)} - a \rVert \to 0 \text{ } (k \to \infty)$
Ist $a=(a_1, \ldots, a_n) \in \mathbb{R}^n)$, so gilt: $a^{(k)} \to a \text{ } (k \to \infty) \Leftrightarrow \forall j \in \{ 1, \ldots, n\}:$
$a_j^{(k)} \to a_j \text{ } (k \to \infty)$ Konvergenz im $\mathbb{R}^n$ ist also gleichbedeutend mit \underline{koordinatenweiser Konvergenz}.
Sonst vieles wie in HM1 (Bolzano-Weierstra0, Cauchy-Kriterium,...)

\section{Grenzwerte bei Funktionen, Stetigkeit}
Definition: f heißt in $x_0 \in D \text{ stetig }: \Leftrightarrow$ Für jede Folge $(x^{(k)})$ in $D$ mit $x^{(k)} \to x_0$ gilt: 
$f(x^{(k)}) \to f(x_0)$ \\
\begin{align*}
    f \text{ ist auf } D \text{ stetig } \Leftrightarrow f \text{ ist in jedem } x \in D \text{ stetig }
\end{align*}
Sonst vieles wie in HM1($C(D,\mathbb{R}^m)$ ist ein Vektorraum, Verkettung stetiger Funktionen ist stetig usw.) \\
Wichtig: Ist eine Funktion $f: \mathbb{R}^n \to \mathbb{R}^m$ linear, so ist sie stetig.

\section{Analysis in $\mathbb{C}$}
Konvergenz von \underline{Folgen}: Es sei $(z_n)$ eine Folge in $\mathbb{C}$ und $z_0 \in \mathbb{C}$. Dann gilt: 
$z_n \to z_0 \Leftrightarrow \lvert z_n - z_0 \rvert \to 0 \Leftrightarrow \text{Re}(z_n) \to \text{Re}(z_0) 
\text{ und Im}(z_n) \to \text{Im}(z_0)$ \\
Definitionen und Sätze über \underline{Reihen} wie in HM1 für Reihen in $\mathbb{R}$ (bis auf die Sätze, in denen die Anordnung auf $\mathbb{R}$
eine Rolle spielt (Monotoniekriterium, Leibnizkriterium)). \underline{Potenzreihen} wie in HM1 für reelle Reihen. 

\subsection{Komplexe Fourierreihen}
Definition Komplexes Integral:
Sei $f(x)=u(x) + iv(x)$. Sind $u,v \in R([a,b],\mathbb{R}$, so schreiben wir $f \in R([a,b],\mathbb{C})$ und definieren
\begin{align*}
    \int \limits_a^b f(x) dx \coloneqq \int \limits_a^b u(x) dx + i \int \limits_a^b v(x) dx
\end{align*}

Definition: Sei $f \in R([-\pi,\pi],\mathbb{C})$. Dann heißen die Zahlen $c_n \coloneqq \frac{1}{2\pi} \int \limits_{-\pi}^{\pi} f(x) e^{-inx} dx$
die \underline{komplexen Fourierkoeffizienten} (FK) von $f$ und $\sum \limits_{n= - \infty}^{\infty} c_n e^{inx}$ heißt
die komplexe Fourierreihe (FR). Schreibweise: $f \sim \sum \limits_{n=- \infty}^{\infty} c_n e^{inx}$ \\
Bemerkung: Ist $f \in R([-\pi, \pi], \mathbb{R}) \text{ und } x \in \mathbb{R}$, so gilt: \\
Die komplexe Fourierreihe konvergiert in $x \Leftrightarrow$ die reelle Fourierreihe konvergiert in $x$.

\section{Differentialrechnung im $\mathbb{R}^n$ (reelwertige Funktionen)}
Definition:
\begin{enumerate}
    \item $f$ heißt in $x_0$ partiell differenzierbar (pdb) nach $x_i$: $\Leftrightarrow$
    Es existiert 
    \begin{align*}
        f_{x_i}(x_0) \coloneqq \frac{\partial f}{\partial x_i}(x_0) \coloneqq \lim \limits_{t \to 0} \frac{f(x_0 + te_i) - f(x_0)}{t} \in \mathbb{R}.
    \end{align*}
    In dem Fall heißt $f_{x_i}(x_0)$ die \underline{partielle Ableitung von $f$ in $x_0$ nach $x$.}
    \item $f$ heißt in $x_0 \in D$ pdb $\Leftrightarrow$ $f$ ist in $x_0$ pdb nach allen Variablen $x_1,\ldots,x_n$.
    I.d.F. heißt der Vektor
    \begin{align*}
        \text{grad} f(x_0) \coloneqq (f_{x_1}(x_0),\ldots,f_{x_n}(x_0))
    \end{align*}
    der \underline{Gradient von $f$ in $x_0$}
    \item Es sei $i \in \{1,\ldots,n\}$ und $f_{x_i}$ sei auf $D$ vorhanden. Also haben wir die partielle Ableitung von $f$ nach $x_i: f_{x_i}: D \to \mathbb{R}$.
    Es sei $x_0 \in D$ und $j \in \{1, \ldots,n \}$. Ist $f_{x_i}$ pdb nach $x_j$, so heißt
    \begin{align*}
        f_{x_ix_j}(x_0) \coloneqq \frac{\partial^2 f}{\partial x_j \partial x_i} (f_{x_i})_{x_j}(x_0)
    \end{align*}
    \underline{die partielle Ableitung 2. Ordnung von $f$ nach $x_i$ und $x_j$}
\end{enumerate}

\subsection{Satz von Schwartz}
\label{sec: Schwartz}
Es sei $m \in \mathbb{N}$ und $f \in C^m(D,\mathbb{R})$. Dann ist jede partielle Ableitung von $f$ der Ordnung $leq m$
unabhängig von der Reihenfolge der Differentiation.

\subsection{Differenzierbarkeit}
Definition: $f$ heißt \underline{in $x_0 \in D$ differenzierbar} (db): $\Leftrightarrow$
\begin{align*}
    \exists a \in \mathbb{R}^n: \lim \limits_{h \to 0} \frac{f(x_0+h) -f(x_0) - a\cdot h}{\lVert h \rVert} = 0 \\
    \Leftrightarrow \exists a \in \mathbb{R}^n : \lim \limits_{x \to x_0} \frac{f(x) -f(x_0) - a\cdot (x-x_0)}{\lVert x-x_0 \rVert} = 0
\end{align*}
\begin{enumerate}[a)]
    \item Ist $f$ in $x_0$ db, so ist $f$ in $x_0$ stetig und $f$ ist in $x_0$ pdb
    \item Ist $f$ in $x_0$ db, so ist der Vektor $a$ in obiger Definition eindeutig bestimmt un $a=\text{grad}f(x_0)$ I.d.F. gilt:
    \begin{align*}
        f'(x_0) \coloneqq a = \text{grad}f(x_0)
    \end{align*}
    \item Es sei $f$ auf $D$ pdb und $f_{x_1},\ldots,f_{x_n}$ seien in $x_0 \in D$ stetig. Dann ist $f$ in $x_0$ db.
\end{enumerate}

\subsection{Kettenregel}
Sei $I \subseteq \mathbb{R}$ ein Intervall, $g: I \to \mathbb{R}^n$ db in $f_0 \in I, g(I) \subseteq D$ und $f$ sei in $x_0 \coloneqq g(t_0)$ db. 
Dann ist $f \circ g: I \to \mathbb{R}$ db in $t_0$ und $(f \circ g)' (t_0) = f'(g(t_0)) \cdot g'(t_0)$, wobei hier das (euklidische) Skalarprodukt gemeint ist.

\subsection{Mittelwertsatz}
Es sei $f D \to \mathbb{R}$ auf $D$ differenzierbar, es seien $a,b \in D$ und $S[a,b] \subseteq D$. Dann existiert ein $\xi \in S[a,b]$ mit $f(b) - f(a) = f'(\xi) \cdot (b-a)$

\subsection{Richtungen}
\begin{enumerate}[a)]
    \item Es sei $a \in \mathbb{R}^n$. Ist $\lVert a \rvert = 1,$ so heißt $a$ \underline{ein(e) Richtung(svektor)}
    \item Sei $x_0 \in D$ und $a \in \mathbb{R}^n$ eine Richtung. $f$ heißt \underline{in $x_0$ in Richtung $a$ differenzierbar} $\Leftrightarrow$ Es existiert der Grenzwert
    \begin{align*}
        \frac{\partial f}{\partial a}(x_0) \coloneqq \lim \limits_{t \to 0} \frac{f(x_0 + ta) -f(x_0)}{t} \in \mathbb{R}.
    \end{align*}
    I.d.F. heißt $\frac{\partial f}{\partial a}(x_0)$ die \underline{Richtungsableitung von $f$ in $x_0$ in Richtung $a$}.
    \item Ist $f$ in $x_0 \in D$ db und $a \in \mathbb{R}^n$ eine Richtung, so existiert $\frac{\partial f}{\partial a}(x_0)$ und 
    $\frac{\partial f}{\partial a}(x_0) = a \cdot \text{grad}f(x_0)$
\end{enumerate}

\subsection{Hesse-Matrix}
Es sei $f \in C^2(D,\mathbb{R})$ und $x_= \in D$. Die Matrix 
\begin{align*}
    H_f(x_0) \coloneqq \begin{pmatrix} f{x_1x_1}(x_0) & \ldots & f{x_1x_n}(x_0) \\ \vdots & & \vdots \\ f{x_nx_1}(x_0) & \ldots & f{x_nx_n}(x_0)\end{pmatrix}
\end{align*}
heißt \underline{Hesse-Matrix von $f$ in $x_0$}. Sie ist nach dem \hyperref[sec: Schwartz]{Satz von Schwartz} symmetrisch.

\subsection{Satz von Taylor}
Sei $f \in C^2(D,\mathbb{R}), x_0 \in D, h \in \mathbb{R}^n$ und $S[x_0,x_0+h] \subseteq D$. Dann existiert $\xi \in S[x_0,x_0+h]$ mit
\begin{align*}
    f(x_0+h) = f(x_0) + \text{ grad}f(x_0) \cdot h + \frac{1}{2} H_f(\xi) \cdot h
\end{align*}

\subsection{Definitheit}

Definition: Sei $A \in \mathbb{R}^{n \times n}$ symmetrisch. $A$ heißt
\begin{enumerate} [a)]
    \item positiv definit $\Leftrightarrow \forall x \in \mathbb{R}^n \setminus \{0\} (Ax) \cdot x > 0$
    \item negativ definit $ \Leftrightarrow \forall x \in \mathbb{R}^n \setminus \{0\} (Ax) \cdot x < 0$
    \item indefinit $\Leftrightarrow \exists u,v \in \mathbb{R}^n: (Au) \cdot u > 0$ und $(Av) \cdot v < 0$
\end{enumerate}
\vspace{0.5cm}

Kriterien für Definitheit:
\begin{enumerate} [a)]
    \item \begin{enumerate} [i)]
        \item $A$ ist positiv definit $\Leftrightarrow$ alle EW von $A$ sind $> 0$
        \item $A$ ist negativ definit $\Leftrightarrow$ alle EQ von $A$ sind $< 0$
        \item $A$ ist indefinit $\Leftrightarrow$ es gibt EW die $> 0$ und welche, die $< 0$ sind
    \end{enumerate}
    \item Sei $n=2, A = \begin{pmatrix} \alpha & \beta \\ \beta & \gamma\end{pmatrix}$
    \begin{enumerate} [i)]
        \item A positiv definit $\Leftrightarrow \alpha > 0, \text{ det } A > 0$
        \item A negativ definit $\Leftrightarrow \alpha < 0, \text{ det } A < 0$
        \item A ist indefinit $\Leftrightarrow \text{det } A < 0$
    \end{enumerate}
\end{enumerate}

\subsection{Lokale Extrema}
\begin{enumerate} [a)]
    \item Ist $f$ in $x_0 \in D$ pdb und hat $f$ in $x_0$ ein lokales Extremum, so ist $\text{grad}f(x_0) = 0.$
    \item Ist $f \in C^2(D,\mathbb{R}), x_0 \in D$ und $\text{grad}f(x_0) = 0$, so gilt:
    \begin{enumerate} [i)]
        \item Ist $H_f(x_0)$ positiv definit, so hat $f$ in $x_0$ ein lokales Minimum
        \item Ist $H_f(x_0)$ negativ definit, so hat $f$ in $x_0$ ein lokales Maximum
        \item Ist $H_f(x_0)$ indefinit, so hat $f$ in $x_0$ kein lokales Extremum
    \end{enumerate}
\end{enumerate}
\underline{Problem:} Überprüfe Menge auf Extrema
\underline{Lösung:} 
\begin{enumerate}
    \item Überprüfe, ob Menge überhaupt Extrempunkte haben kann/muss ($\widehat{=}$ dass sie kompakt ist)
    \item Überprüfe das innere der Menge auf Extrema
    \item Überprüfe Ränder der Menge auf Extram, sofern diese vorhanden sind
\end{enumerate}

\section{Differentialrechnung im $\mathbb{R}^n$ (vektorwertige Funktionen)}
\underline{Definition:}
\begin{enumerate}
    \item Es sei $x_0 \in D$. $f$ heißt in $x_0$ partiell differenzierbar $\Leftrightarrow$ Alle $f_j$ sind in $x_0$ pdb. In diesem Fall heißt
    \begin{align*}
        \frac{\partial f}{\partial x}(x_0) \coloneqq \frac{\partial (f_1, \ldots, f_n)}{\partial (x_1, \ldots, x_n)} (x_0) \coloneqq J_f(x_0) \coloneqq
        \begin{pmatrix} \frac{\partial f_1}{\partial x_1}(x_0) & \ldots & \frac{\partial f_1}{\partial x_n}(x_0) \\ \vdots & & \vdots \\ 
        \frac{\partial f_n}{\partial x_1}(x_0) & \ldots & \frac{\partial f_n}{\partial x_n}(x_0)\end{pmatrix}
    \end{align*}
    die \underline{Jacobi-} oder \underline{Fundamentalmatrix von $f$} in $x_0$. In der Jacobimatrix stehen zeilenweise die Gradienten der Koordinatenfunktionen!
    \item $f$ heißt in $x_0 \in D$ db: $\Leftrightarrow$ Es existiert eine $m \times n$-Matrix $A$ mit: 
    \begin{align*}
        \lim \limits_{h \to 0} \frac{f(x_0 + h) - f(x_0)}{\lVert h \rVert} = 0
    \end{align*}
\end{enumerate}

\subsection{Ableitungsbegriff und Folgerungen}
\begin{enumerate} [a)]
    \item $f$ ist in $x_0$ db $\Leftrightarrow$ Alle $f_j$ sind in $x_0$ db. I.d.F. gilt:
    \begin{enumerate} [i)]
        \item $f$ ist stetig in $x_0$
        \item $f$ ist in $x_0$ pdb
    \end{enumerate}
    \item Ist $f$ in $x_0$ db, so heißt $f'(x_0) \coloneqq J_f(x_0)$ die \underline{Ableitung von $f$ in $x_0$.}
    \item Sind alle partiellen Ableitungen $\frac{\partial f_i}{\partial x_k}$ vorhanden auf $D$ und in $x_0$ stetig, so ist $f$ in $x_0$ db.
\end{enumerate}

\subsection{Kettenregel}
Sei $f: D \to \mathbb{R}^m$ in $x_0 \in D$ db und $\Tilde{D} \subseteq \mathbb{R}^m$, offen $f(D) \subseteq \Tilde{D}$ und $g: \Tilde{D} \to \mathbb{R}^p$ db
in $y_0 \coloneqq f(x_0)$. Dann ist  $\phi \coloneqq g\circ f: D \to \mathbb{R}^p$ in $x_0$ db und 
\begin{align*}
    \phi'(x_0) = (g \circ f)'(x_0) = \underbrace{g'(f(x_0)) \cdot f'(x_0)}_{\text{Matrizenprodukt}}
\end{align*}

\subsection{Implizit definierte Funktionen}
\underline{Spezialfall:} Seien $n=2, f\in C^1(D, \mathbb{R}),(x_0,y_0) \in D, f(x_0,y_0)=0$ und $f_y(x_0,y_0) \neq 0$. Dann existieren $\delta,\eta > 0$ und genau eine
stetig differenzierbare Funktion $g: (x_0- \delta x_0 + \delta) \to (y_0 - \eta, y_0 + \eta)$ mit $g(x_0)=y_0$
und $\forall x \in (x_0- \delta x_0 + \delta): f(x,g(x))=0$ \\
\underline{Satz über implizit definierte Funktionen} \\
Es sei $(x_=,y_0) \in D, f(x_0,y_0) = 0$ und det $\frac{\partial f}{\partial y}(x_0,y_0) \neq 0$ (wichtig!). Dann existieren $\delta,\eta > 0$ mit folgenden
Eigenschaften:
\begin{enumerate} [a)]
    \item $U_\delta (x_0) \times U_\eta (y_0) \subseteq D$
    \item $\forall x \in U_\delta(x_0) \exists_1 y \eqqcolon g(x) \in U_\eta(y_0): f(x,y)=0$
    \item $g \in C^1(U_\delta(x_0),\mathbb{R}^p)$
    \item $\forall x \in U_\delta(x_0): \text{det} \frac{\partial f}{\partial y}(x,g(x)) \neq 0$
    \item $\forall x \in U_\delta(x_0): g'(x)=-(\frac{\partial f}{\partial y}(x,g(x))^{-1} \cdot (\frac{\partial f}{\partial x}(x,g(x))$
\end{enumerate}

\subsection{Umkehrsatz}
Es sei $D \subseteq \mathbb{R}^n$ offen, $f \in C^1(D, \mathbb{R}^n)$ und $x_0 \in D$. Ist $\text{det}f'(x_0) \neq 0$, so existiert ein
$\delta > 0$ mit:
\begin{enumerate} [a)]
    \item $U_\delta(x_0) \subseteq D$ und $f(U_\delta(x_0))$ ist offen
    \item $f$ ist auf $U_\delta(x_0)$ injektiv
    \item $f^{-1}: f(U_\delta(x_0)) \to U_\delta(x_0)$ ist in $C^1(f(U_\delta(x_0)),\mathbb{R}^n), \text{det}f'(x) \neq 0 (x \in U_\delta(x_0))$ und
    \begin{align*}
        (f^{-1})'(y) = (f'(f^{-1}(y)))^{-1} \hspace{0.5cm} (y \in f(U_\delta(x_0))
    \end{align*}
\end{enumerate}

\section{Integration im $\mathbb{R}^n$}
\subsection{Satz von Fubini}
Es seien $p,q \in \mathbb{N}, n=p+q$. Es sei $I_1$ ein kompaktes Intervall im $\mathbb{R}^p$, $I_2$ sei ein kompaktes Intervall im $\mathbb{R}^q$,
es sei $I \coloneqq I_1 \times I_2 \subseteq \mathbb{R}^n$ und $f \in R(I)$.
\begin{enumerate} [a)]
    \item Für jedes feste $y \in I_2$ ist die Funktion $x \mapsto f(x,y)$ integrierbar über  $I_1$ und es sei 
    $g(y) \coloneqq \int_{I_1} f(x,y) dx$. Dann gilt $g \in R(I_2)$ und
    \begin{align*}
        \int_I f(x,y) d(x,y) = \int_{I_2} g(y) dy = \int_{I_2} (\int_{I_1} f(x,y) dx) dy
    \end{align*}
    \item Für jedes feste $x \in I_1$ ist die Funktion $y \mapsto f(x,y)$ integrierbar über  $I_2$ und es sei 
    $g(x) \coloneqq \int_{I_2} f(x,y) dy$. Dann gilt $g \in R(I_1)$ und
    \begin{align*}
        \int_I f(x,y) d(x,y) = \int_{I_1} g(x) dx = \int_{I_1} (\int_{I_2} f(x,y) dy) dx
    \end{align*}
\end{enumerate}
\underline{Folgerung:} Es sei $I = [a_1,b_1] \times \ldots \times [a_n,b_n]$ und $f \in C(I)$. Dann ist
\begin{align*}
    \int_I f(x) dx = \int_I f(x_1, \ldots, x_n) d(x_1, \ldots, x_n) = \int \limits_{a_1}^{b_1} (\ldots \int \limits_{a_{n-1}}^{b_{n-1}}(\int \limits_{a_n}^{b_n}
    f(x_1,\ldots,x_n)dx_n)dx_{n-1}\ldots) dx_1
\end{align*}
wobei die Reihenfolge der Integration \underline{beliebig vertauscht} werden kann.

\subsection{Prinzip von Cavalieri}
\label{sec: Cavalieri}
\underline{Voraussetzung:} Es sei $B \subseteq \mathbb{R}^{n+1}$ messbar. Für Punkte im $\mathbb{R}^{n+1}$ schreiben wir $(x,z)$ mit $x \in \mathbb{R}^n$ und $z \in \mathbb{R}$. 
Es seien $a,b \in \mathbb{R}$ so, dass $a \leq z \leq b \text{ } ((x,z) \in B)$. Für $z \in [a,b]$ sei 
\begin{align*}
    Q(z) \coloneqq \{ x \in \mathbb{R}^n: (x,z) \in B\}
\end{align*}
Weiter sei $Q(z)$ messbar für jedes $z \in [a,b]$. Dann ist $z \mapsto \lvert Q(z)\rvert$ integrierbar über $[a,b]$ und 
\begin{align*}
    \lvert B\rvert = \int \limits_a^b \lvert Q(z)\rvert dz
\end{align*}
\underline{Rotationskörper:} Sei $a < b, f \in R([a,b])$ und $f \geq 0$ auf $[a,b]$. Der Graph von $f$ rotiere z.B. um die $x$-Achse:
\begin{align*}
    B = \{ (x,y,z) \in \mathbb{R}^3: y^2+z^2 \leq f(x)^2\}
\end{align*}
Für $x \in [a,b]$ ist dann $Q(x) = \{ (y,z) \in \mathbb{R}^2: y^2+z^2\leq f(x)^2\}$. Also gilt: $\lvert Q(x) \rvert = \pi f(x)^2$ und somit $\lvert B \rvert = \pi \int \limits_a^b f(x)^2 dx$ \\
\underline{Normalbereich bzgl. $x$-Achse:} \\
Sei $a,b \in \mathbb{R}, a < b, f,g \in C([a,b])$ und $f \leq g$ auf $[a,b]$. Dann heißt die Menge
\begin{align*}
    B \coloneqq \{ (x,y) \in \mathbb{R}^2: x \in [a,b], f(x) \leq y \leq g(x)\}
\end{align*}
ein \underline{Normalbereich bzgl. der $x$-Achse.} Es gilt:
\begin{align*}
    \int_B h(x,y) d(x,y) = \int_I h_B(x,y) d(x,y) = \int \limits_a^b (\int \limits_{f(x)}^{g(x)} h(x,y) dy)dx
\end{align*}
\underline{Normalbereich bzgl. der $y$-Achse:} \\
Seien $a,b,f$ und $g$ wie in obiger Definition. Dann heißt die Menge 
\begin{align*}
    B \coloneqq \{ (x,y) \in \mathbb{R}^3: y \in [a,b], f(y) \leq x \leq g(y)\}
\end{align*}
ein \underline{Normalbereich bzgl. der $y$-Achse}. Wie oben gilt:
\begin{align*}
    \int_B h(x,y) d(x,y) = \int \limits_a^b (\int \limits_{f(y)}^{g(y)} h(x,y) dx)dy
\end{align*}

\subsection{Substitutionsregel}
\label{sec: Sub}
Es sei $G \subseteq \mathbb{R}^n$ offen, $g \in C^1(G,\mathbb{R}^n)$ und $B \subseteq G$ kompakt und messbar. Weiter sei $g$ auf dem Inneren $B^\circ$ von $B$ injektiv
und det$g'(y) \neq 0 \text{ } (y \in B^\circ)$. Ist dann $A \coloneqq g(B)$ und $f \in C(A,\mathbb{R)}$, so ist $A$ kompakt und messbar und es gilt:
\begin{align*}
    \int_A f(x) dx = \int_B f(g(y)) \lvert \text{det} g'(y)\rvert dy
\end{align*}
\underline{Polarkoordinaten (n=2):} \\
$x = r\cos \varphi, y = r\sin \varphi \text{ } (r = \lVert(x,y)\rVert = \sqrt{x^2 + y^2})$
\begin{align*}
    g(r,\varphi) \coloneqq (r\cos \varphi, r\sin \varphi), \det g'(r,\varphi) = r \text{ } ((r,\varphi) \in G \coloneqq \mathbb{R}^2)
\end{align*}
(Typische Menge: $A=\{ (x,y) \in \mathbb{R}^2: 0 \leq x^2+y^2\leq R^2\} (R \in \mathbb{R})$) \\
Betrachte $0 \leq \varphi_1 < \varphi_2 \leq 2\pi, 0 \leq R_1 < R_2$
\begin{align*}
    A \coloneqq \{ (r\cos \varphi, r\sin \varphi): \varphi \in [\varphi_1,\varphi_2], r \in [R_1,R_2]\}
\end{align*}
Mit $B \coloneqq [R_1,R_2] \times [\varphi_1, \varphi_2]$ ist $A = g(B)$. Auf $B^\circ = (R_1,R_2) \times (\varphi_1,\varphi_2)$ ist $g$ injektiv und $\det g' \neq 0$.
Ist nun $f \in C(A,\mathbb{R}$, so gilt: 
\begin{align*}
    \int_A f(x,y) d(x,y) = \int_B f(r \cos \varphi,r \sin \varphi) \cdot \underbrace{r}_{= \lvert \det g'(r,\varphi)\rvert} d(r, \varphi) \stackrel{\text{Fubini}}{=}
    \int \limits_{\varphi_1}^{\varphi_2} (\int \limits_{R_1}^{R_2} f(r \cos \varphi, r \sin \varphi)r dr) d\varphi
\end{align*}
\underline{Zylinderkoordinaten (n=3):} \\
\begin{align*}
    \begin{rcases} 
        x = r \cos \varphi \\
        y = r \sin \varphi \\
        z = z \\
    \end{rcases} g(r,\varphi,z) \coloneqq (r \cos \varphi,r \sin \varphi, z) \det g'(r,\varphi,z) = r
\end{align*}
(Typische Menge: $ \{ (x,y,z) \in \mathbb{R}^3: x^2+y^2 \leq R^2, 0 \leq z \leq h\} \text{ } (R,h \in \mathbb{R}_+)$)\\
Es seien $A,B \subseteq \mathbb{R}^3$ wie bei der \hyperref[sec: Sub]{Substitutionsregel} und $f \in C(A,\mathbb{R})$. Dann gilt: 
\begin{align*}
    \int_A f(x,y,z) d(x,y,z) = \int_B f(r \cos \varphi, r \sin \varphi, z) \cdot r d(r,\varphi,z)
\end{align*}
\underline{Kugelkoordinaten (n=3):} \\
Für $\varphi = [0,2\pi], \vartheta \in [-\frac{\pi}{2}, \frac{\pi}{2}], r=\lVert (x,y,z)\rVert = \sqrt{x^2+y^2+z^2}, x= r\cos \varphi \cos \vartheta, y=r \sin \varphi \cos \vartheta$,
$z=r \sin \vartheta$,
\begin{align*}
    g(r,\varphi,\vartheta) \coloneqq (r\cos \varphi \cos \varphi, r \sin \varphi \cos \vartheta, r \sin \vartheta), \lvert \det g'(r,\varphi,\vartheta)\rvert = r^2 \cos \vartheta
\end{align*}
Sind $A,B \subseteq \mathbb{R}^3$ wie beim \hyperref[sec: Cavalieri]{Prinzip von Cavalieri} (also $A=g(B)$), so gilt für $f \in C(A,\mathbb{R})$:
\begin{align*}
    \int_A f(x,y,z) d(x,y,z) = \int_B f(g(r,\varphi,\vartheta)) \cdot r^2 \cos \vartheta d(r,\varphi,\vartheta)
\end{align*}
Typische Menge: $\{ (x,y,z) \in \mathbb{R}^3: x,y,z \geq 0, x^2+y^2+z^2 \leq 1\}$

\section{Spezielle Differentialgleichungen 1. Ordnung}
\underline{Definition:} Sei $\emptyset \neq D \subseteq \mathbb{R}^3$. Die Gleichung
\begin{align*}
    f(x,y(x),y'(x)) = 0
\end{align*}
heißt \underline{Differentialgleichung (Dgl) 1. Ordnung}. Sind $x_0,y_0 \in \mathbb{R}$, so heißt
\begin{align*}
    (A) \begin{cases}
    f(x,y(x),y'(x)) = 0 \\
    y(x_0) = y_0
    \end{cases}
\end{align*}
ein \underline{Anfangswertproblem (Awp)}.

\subsection{Dgl mit getrennten Veränderlichen}
Es seien $I_1, I_2 \subseteq \mathbb{R}$ Intervalle, es seien $f \in C(I_1,\mathbb{R})$ und $g \in C(I_2,\mathbb{R})$. Die Dgl
\begin{equation} \label{dgl_getrennt}
    y'(x) = f(x)g(y(x))
\end{equation}
heißt \underline{Dgl mit getrennten Veränderlichen}. Ist $g(y) \neq 0 \text{ } (y \in I_2)$, so erhält man von (\ref{dgl_getrennt}), indem man die Gleichung
$\int \frac{dy}{g(y)} = \int f(x) dx + c$ nach $y$ auflöst.
\underline{Merkregel:}
\begin{align*}
    y' = f(x)g(y) \Rightarrow \frac{dy}{dx} = f(x)g(y) \Rightarrow \frac{dy}{g(y)} = f(x) dx \Rightarrow \int \frac{dy}{g(y)} = \int f(x) dx + c
\end{align*}
Lösung des Awp: In die vorhandene Gleichung $x_0$ einsetzen und nach $c$ auflösen.

\subsection{Lineare Dgls}
\begin{equation} \label{ldgl_eq}
    y'(x) = \alpha(x) y(x) + s(x) 
\end{equation}
heißt \underline{lineare Dgl} und $s$ heißt \underline{Störfunktion}. $y'(x)=\alpha(x)y(x)$ heißt die zu (\ref{ldgl_eq}) gehörige \underline{homogene Gleichung}. \\
\underline{Lösen von Linearen Dgls:} \\
Sei $B$ eine Stammfunktion von $\alpha$ auf $I$
\begin{enumerate} [a)]
    \item Sei $y: I \to \mathbb{R}$ eine Funktion. Dann gilt:
    \begin{enumerate} [i)]
        \item $y$ ist eine Lösung der homogenen Gleichung auf $I \Leftrightarrow \exists c \in \mathbb{R}: y(x) = ce^{\beta(x)}$
        \item Sei $y_p$ eine spezielle Lösung von (\ref{ldgl_eq}) auf $I$. Dann gilt: $y ist eine$ Lösung von (\ref{ldgl_eq}) auf $I$
        $\Leftrightarrow \exists c \in \mathbb{R}: y(x) = y_p(x) + ce^{\beta(x)}$
    \end{enumerate}
    \item Variation der Konstanten: Der Ansatz 
    \begin{align*}
        y_p(x) = c(x)e^{\beta(x)}
    \end{align*}
    mit $c(x)$ unbekannt führt auf eine Lösung.
\end{enumerate}
\underline{Vorgehen:} \\
Gegeben: Gleichung $y'(x) = \alpha(x)y(x)$
\begin{enumerate}
    \item Lösung der homogenen Gleichung: Setze $y(x) = ce^{\beta(x)}$, wobei $\beta$ Stammfunktion von $\alpha$ ist.
    \item Lösung der inhomogenen Gleichung: \\
    Ansatz: $y_p'(x) = c'(x)e^{\beta(x)} + c(x) e^{\beta(x)}\beta'(x) \stackrel{\text{!}}{=} y_p(x)\alpha(x) + s(x)$ \\
    Löse diese Gleichung nach $c(x)$ auf, dann $y_p(x) = c(x)e^{\beta(x)}$. Allgemeine Lösung: $y(x)=ce^{\beta(x)} + y_p(x)$ \\
    Awp lösen wie immer
\end{enumerate}

\subsection{Bernoulli- und Riccatti-Differentialgleichungen}
\begin{align*}
    y'(x) + g(x)y(x) + h(x)(y(x))^\alpha = 0
\end{align*}
ist eine \underline{Bernoulli Dgl}. \\
Lösung: Sei $\alpha \in \mathbb{R} \setminus \{0,1\}$. Betrachte Transformation $z(x) = (y(x))^{1 - \alpha}$:
\begin{align*}
    z'(x) = - (1-\alpha)g(x)z(x) -(1-\alpha)h(x)
\end{align*}
Löse die lineare DGL für $z$. Ist $z$ eine Lösung der Dgl, dann setze $y(x) \coloneqq z(x)^{\frac{1}{1 - \alpha}}$ für $x$ aus $I_1 \subseteq I$,
für das $(z(x))^{\frac{1}{1- \alpha}}$ eine differenzierbare Funktion liefert, Dann ist $y$ eine Lösung auf $I_1$.
\begin{equation} \label{eq_ricatti}
    y'(x) + g(x)y(x) + h(x)y^2(x) = k(x)
\end{equation}
heißt \underline{Riccatische Dgl}. Sind $y_1,y_2$ Lösungen von (\ref{eq_ricatti}) auf $I_1 \subseteq I$, so gilt für $u \coloneqq y_1 - y_2$:
\begin{align*}
    u'(x) = -(g(x) + 2h(x)y_2(x))u(x) - h(x)u^2(x)
\end{align*}
Fazit: Ist eine Lösung $u \neq 0$ von (\ref{eq_ricatti}) bekannt, so liefern Lösungen $u \neq 0$ obiger Bernoulli-Dgl für $u$ weitere Lösungen der Form $y_2(x) + u(x)$.

\subsection{Lineare Systeme mit konstanten Koeffizienten}
\underline{Definition:}
\begin{align*}
    y'(x) = Ay(x) + b(x)
\end{align*}
heißt das \underline{homogene} (wenn $b(x) = 0$) bzw. das \underline{inhomogene System} (wenn $b(x) \neq 0)$. \\
\underline{Lösungsmethode:}
\begin{enumerate}
    \item Alle Eigenwerte (auch komplexe) bestimmen als Nullstellen des charakteristischen Polynoms.
    \item Fasse in einer Menge $M$ alle reellen Eigenwerte und bei den komplexen Eigenwerten nur den Eigenwert selbst und \textbf{nicht} sein komplex Konjugiertes zusammen.
    \item Bestimme eine Basis des Hauptraums für jedes Element der Menge $M$
    \item Für jeden Basisvektor $v$ jedes Elements $\lambda_j$ aus $M$, setze:
    \begin{align*}
        y(x) \coloneqq e^{\lambda_jx} \cdot (v+\frac{x}{1!}(A-\lambda_jI)v+ \ldots + \frac{x^{k_j-1}}{(k_j-1)!}(A-\lambda_jI)^{k_j-1}v) \\
        (k_j  \text{ entspricht der algebraischen Vielfachheit von }\lambda_j)
    \end{align*}
    Fall 1: Ist $\lambda_j$ reell, so ist $y(x)$ eine Lösung der Dgl auf $\mathbb{R}$ \\
    Fall 2: Ist $\lambda_j$ komplex, so ist $y(x)$ auch komplex. Dann sind Real- und Imaginärteil von $y(x)$ jeweils linear unabhängige Lösungen der Dgl
    \item Fasse alle $y(x)$ in einer Menge zusammen. Nun hat man das zur Dgl gehörige Fundamentalsystem.
\end{enumerate}
\underline{Awp Lösen:} Lgs lösen \\
\underline{Inhomogene Gleichung lösen:} \\
Ansatz: 
\begin{align*}
    y_p(x) = \underbrace{Y(x)}_{\text{Fundamentalmatrix}}c(x) 
\end{align*}
wobei $c$ noch unbekannt. Dann gilt: $y_p$ ist eine Lösung auf $I \Leftrightarrow c'(x)=(Y(x))^{-1}b(x)$. Bestimme $c(x)$ und erhalte damit $y_p$.
Dann ist 
\begin{align*}
    y(x) = Y(x) \begin{pmatrix} c_1 \\ \vdots \\ c_n\end{pmatrix} + y_p(x)
\end{align*}
eine allgemeine Lösung

\section{Lineare Dgl n-ter Ordnung mit konstanten Koeffizienten}
Sei 
\begin{align*}
    Ly \coloneqq y^{(n)} + a_{n-1}y^{(n-1)}+ \ldots + a_1y' + a_0y
\end{align*}
Die Dgl $Ly(x)=b(x)$ heißt lineare Dgl n-ter Ordnung mit konstanten Koeffizienten. \\
Lösungsmethode für die homogene Gleichung ($Ly(x)=0$):
\begin{enumerate}
    \item $p(\lambda)=\lambda^n + a_{n-1}\lambda^{n-1} + \ldots + a_1\lambda + a_0$ heißt charakteristisches Polynom der homogenen Dgl.
    Bestimme dessen Nullstellen.
    \item Fasse alle reellen Nullstellen und alle komplexen ohne ihr komplex Konjugiertes in einer Menge $M$ zusammen.
    \item FÜr jedes $\lambda_j$ aus $M$ mit Vielfachheit $k_j$: \\
    Fall 1: Ist $\lambda_j$ reell, so sind
    \begin{align*}
        e^{\lambda_jx}, xe^{\lambda_jx}, \ldots, x^{k_j-1}e^{\lambda_jx}
    \end{align*}
    $k_j$ linear unabhängige Lösungen der Dgl. \\
    Fall 2: $\lambda_j = \alpha+i\beta \in \mathbb{C} \setminus \mathbb{R}$. Dann sind
    \begin{align*}
        e^{\alpha x}\cos \beta x, \ldots, x^{kj-1}e^{\alpha x}\cos \beta x \\
        e^{\alpha x}\sin \beta x, \ldots, x^{kj-1}e^{\alpha x}\sin \beta x
    \end{align*}
    $2k_j$ linear unabhängige Lösungen der Dgl
    \item Führt man 3. für alle Elemente aus $M$ durch, so erhält man ein Fundamentalsystem zur Dgl.
\end{enumerate}
\underline{Lösung der inhomogenen Gleichung ($Ly(x)=b(x)$:} \\
Seien $\gamma,\delta \in \mathbb{R}, m \in \mathbb{N}_0,q$ ein Polynom vom Grad m und $b$ von der Form $b(x)=q(x)e^{\gamma x}\cos(\delta x)$
oder $b(x)=q(x)e^{\gamma x}\sin(\delta x)$. Sei $p$ das charakteristische Polynom von $Ly(x)=0$. \\
Fall 1: $p(\gamma + i\delta) \neq 0$. Ansatz:
\begin{align*}
    y_p(x) \coloneqq (\widehat{q}(x)\cos(\delta x) + \Tilde{q}(x)\sin(\delta x))e^{\gamma x}
\end{align*}
Fall 2: $\gamma + i \delta$ ist eine $v$-fache Nullstelle von $p$. Ansatz:
\begin{align*}
    y_p(x) \coloneqq x^v(\widehat{q}(x)\cos(\delta x) + \Tilde{q}(x) \sin(\delta x))e^{\gamma x}
\end{align*}
$\widehat{q}$ und $\Tilde{q}$ haben dabei den gleichen Grad wie $q(x)$ oben. \\
Die allgemeine Lösung ist also
\begin{align*}
    y(x) = c_1m_1 + \ldots + c_nm_n + y_p(x)
\end{align*}
für $m_1, \ldots, m_n$ aus dem Fundamentalsystem.

\section{Fouriertransformation}
\underline{Definition:} Eine Funktion heißt auf $[a,b]$ stückweise stetig/glatt $\Leftrightarrow$ Es gibt eine Zerlegung $t_0, \ldots, t_m \in [a,b]$,
sodass $a=t_0 < t_1 < \ldots < t_m=b, \text{ } g \in C((t_{j-1},t_j)), j \in \{ 1, \ldots,m\}$ und es existieren
\begin{enumerate}
    \item $g(a+),g(b-),g(t_j\pm) \text{ } j \in \{ 1, \ldots, m-1\}$ (stückweise stetig)
    \item $g'(t_j \pm), g'(a+),g'(b-)$ (stückweise glatt)
\end{enumerate}
Sei $g$ stückweise glatt. Dann:
\begin{align*}
    g'(x_0) \coloneqq \frac{1}{2} (g'(x_0-) + g'(x_0+))
\end{align*}
\underline{Definition:} $\int \limits_{-\infty}^{\infty} f(t)dt$ heißt (absolut) konvergent $\Leftrightarrow \int \limits_{-\infty}^{\infty} \text{Re}f(t)dt$ 
\underline{und} $\int \limits_{-\infty}^{\infty} \text{Im}f(t)dt$ sind (absolut) konvergent. \\
Im Konvergenzfall: 
\begin{align*}
    \int \limits_{-\infty}^\infty f(t) dt \coloneqq  \int \limits_{-\infty}^\infty \cRe f(t) dt +  \ci \int \limits_{-\infty}^\infty \cIm f(t) dt
\end{align*} 
Ist $ \int \limits_{-\infty}^\infty f(t) dt$ absolut konvergent, so heißt $f$ \underline{absolut integrierbar (aib)}. \\
\underline{Fouriertransformierte:}
Zu einer Funktion $f: \mathbb{R} \to \mathbb{C}$ heißt $\widehat{f}:\mathbb{R} \to \mathbb{C}$ definiert durch:
\begin{align*}
    \widehat{f}(s) \coloneqq \frac{1}{2\pi} \int \limits_{-\int}^\int f(t) e^{-ist} dt
\end{align*}
die Fouriertransformierte von $f$. Die Zuordnung $f \mapsto \widehat{f}$ heißt \underline{Fouriertransformation}.

\subsection{Der Cauchysche Hauptwert}
Definition: $ \int \limits_{-\infty}^\infty f(x) dx= \lim \limits_{\beta \to - \infty} \int \limits_\beta^0 f(x) dx + \lim \limits_{\alpha \to \infty} \int \limits_0^\alpha f(x) dx$ \\
Existiert $\lim \limits_{\alpha \to \infty} \int \limits_{-\alpha}^\alpha f(x) dx$ so heißt diese Zahl \underline{Cauchyscher Hauptwert (CH)} und man schreibt: 
\begin{align*}
    \text{CH- } \int \limits_{- \infty}^\infty f(x) dx \coloneqq \lim \limits_{\alpha \to \infty} \int_{-\alpha}^\alpha f(x) dx
\end{align*}
Ist $\int \limits_{-\infty}^{\infty} f(x) dx$ konvergent, so ist $\text{CH-} \int \limits_{-\infty}^\infty f(x) dx \coloneqq \lim \limits_{\alpha \to \infty} \int \limits_{-\alpha}^\alpha f(x) dx$.
Sei $f: \mathbb{R} \to \mathbb{C}$ stückweise glatt und aib. Dann gilt:
\begin{align*}
    \forall t \in \mathbb{R}: \text{CH-}\int \limits_{-\infty}^\infty \widehat{f}(s)e^{ist} ds = \frac{1}{2} (f(t+) + f(t-))
\end{align*}
\underline{Nützliches zur Fouriertransformation:}
\begin{enumerate} [a)]
    \item $\widehat{\alpha f + \beta g} = \alpha \widehat{f} + \beta \widehat{g}$
    \item Es sei $V \coloneqq \{ f: \mathbb{R} \to \mathbb{C}: f \text{ ist stückweise stetig und absolut integrierbar}\}$. Für jedes $f \in V$ existiert die Fouriertransformierte
    \item Sei $f_h(t) \coloneqq f(t+h)$. Dann ist $f_h \in V$ und $\widehat{f_h}(s) = e^{ish} \widehat{f}(s) \text{ } (s \in \mathbb{R})$
    \item Sei $f: \mathbb{R} \to \mathbb{C}$ stückweise glatt, stetig und aib. Weiter sei $f'$ aib. Dann ist $f' \in V$ und $\widehat{f'}(s) = is\widehat{f}(s)$
\end{enumerate}

\subsection{Faltungen}
\underline{Definition:} Seien $f_1,f_2: \mathbb{R} \to \mathbb{C}$ Funktionen, sodass $\int \limits_{-\infty}^\infty f_1(t-x)f_2(x)dx$ für jedes $t \in \mathbb{R}$ konvergent ist. Dann heißt
\begin{align*}
    f_1 * f_2: \mathbb{R} \to \mathbb{C}, (f_1*f_2)(t) \coloneqq \frac{1}{2\pi} \int \limits_{-\infty}^\infty f_1(t-x)f_2(x)dx
\end{align*}
die \underline{Faltung} von $f_1$ und $f_2$. \\
\underline{Wichtiges zu Faltungen:} Seien $f_1,f_2: \mathbb{R} \to \mathbb{C}$ stetig und aib und $f_1$ beschränkt.
\begin{enumerate} [a)]
    \item $\forall t \in \mathbb{R}: \int \limits_{-\infty}^\infty f_1(t-x)f_2(x)dx$ konvergiert absolut
    \item $f_1*f_2$ ist stetig und aib und
    \begin{align*}
        (\widehat{f_1*f_2})(s) = \widehat{f_1}(s) \widehat{f_2}(s)
    \end{align*}
    \item $\lvert f_1*f_2(t)\rvert \leq \frac{1}{2\pi} \sup \limits_{x \in \mathbb{R}} \lvert f_1(x)\rvert \int \limits_{-\infty}^\infty \lvert f_2(x)\rvert dx \text{ } (t \in \mathbb{R})$
\end{enumerate}

\subsection{Schwartzraum}
Eine Funktion $f \in C^\infty(\mathbb{R},\mathbb{C})$ heißt \underline{schnell fallend} $\Leftrightarrow \forall m,n \in \mathbb{N}_0: t \mapsto t^mf^{(n)}(t)$ ist beschränkt auf $\mathbb{R}$
\begin{align*}
    S \coloneqq \{ f:\mathbb{R} \to \mathbb{C}: f \text{ ist schnell fallend}\}
\end{align*}
heißt \underline{Schwartz-Raum}. \\
\underline{Nützliches zu Funktionen des Schwartz-Raumes:}
Seien $f,g \in S$ und $p$ ein Polynom. Dann:
\begin{enumerate} [a)]
    \item $f$ ist aib und $\lim \limits_{t \to \pm \infty} f(t) = 0$
    \item $S$ ist ein Vektorraum $(\alpha f + \beta g \in S)$
    \item $fg,pf,\overline{f},\cRe f,\cIm f,t \mapsto f(-t) \in S$
    \item $\widehat{f} \in S$ und $f(t)= \int \limits_{-\infty}^\infty \widehat{f}(s)e^{ist}ds \text{ } (t \in \mathbb{R})$
    \item $f^{(n)} \in S \text{ } (n \in \mathbb{N})$ und $\widehat{f^{(n)}}(s) = (is)^n \widehat{f}(s)$
    \item $f*g \in S$ und $\widehat{f*g} = \widehat{f} \cdot \widehat{g}$
    \item Für $h(t) \coloneqq e^{-\frac{t^2}{2}} \text{ } (t \in \mathbb{R})$ gilt: $h \in S$ und $\widehat{h}= \frac{1}{\sqrt{2\pi}}h$ auf $\mathbb{R}$
\end{enumerate}
Die Fouriertransformation $f \mapsto \widehat{f}$ ist ein Isomorphismus von $S$ nach $S$!

\end{document}
